\documentclass{article}

\usepackage{tabularx}
\usepackage{booktabs}
\usepackage{graphicx}
\usepackage{subfig}
\usepackage{subcaption}
\usepackage{fullpage}
\usepackage[round]{natbib}
\usepackage{multirow}
\usepackage{booktabs}
\usepackage{tabularx}
\usepackage{graphicx}
\usepackage{float}
\usepackage{hyperref}

\title{Reflection Report on Bridge Corrosion}

\author{Cynthia Liu}

\date{}

%% Comments

\usepackage{color}

\newif\ifcomments\commentstrue %displays comments
%\newif\ifcomments\commentsfalse %so that comments do not display

\ifcomments
\newcommand{\authornote}[3]{\textcolor{#1}{[#3 ---#2]}}
\newcommand{\todo}[1]{\textcolor{red}{[TODO: #1]}}
\else
\newcommand{\authornote}[3]{}
\newcommand{\todo}[1]{}
\fi

\newcommand{\wss}[1]{\authornote{blue}{SS}{#1}} 
\newcommand{\plt}[1]{\authornote{magenta}{TPLT}{#1}} %For explanation of the template
\newcommand{\an}[1]{\authornote{cyan}{Author}{#1}}

%% Common Parts

\newcommand{\progname}{ProgName} % PUT YOUR PROGRAM NAME HERE
\newcommand{\authname}{Team \#, Team Name
\\ Student 1 name
\\ Student 2 name
\\ Student 3 name
\\ Student 4 name} % AUTHOR NAMES                  

\usepackage{hyperref}
    \hypersetup{colorlinks=true, linkcolor=blue, citecolor=blue, filecolor=blue,
                urlcolor=blue, unicode=false}
    \urlstyle{same}
                                


\begin{document}

\maketitle

%\plt{Reflection is an important component of getting the full benefits from a learning experience.  Besides the intrinsic benefits of reflection, this document will be used to help the TAs grade how well your team responded to feedback.  In addition, several CEAB (Canadian Engineering Accreditation Board) Learning Outcomes (LOs) will be assessed based on your reflections.}

\section{Changes in Response to Feedback}
Feedbacks from Dr. Smith and peer reviewers are reflected in the documents.
%\plt{Summarize the changes made over the course of the project in response to feedback from TAs, the instructor, teammates, other teams, the project supervisor (if present), and from user testers.}

%\plt{For those teams with an external supervisor, please highlight how the feedback from the supervisor shaped your project.  In particular, you should highlight the supervisor's response to your Rev 0 demonstration to them.}

\subsection{SRS and Hazard Analysis}
There are not siginificant changes to the \href{https://github.com/CynthiaLiu0805/BridgeCorrosion/blob/main/docs/SRS/SRS.pdf}{SRS}, however, many details are added or justified. 

\subsubsection{System Context}
In the initial SRS, only user input was considered in the system context section. However, it has now been updated to include developer input as well. This change was made because to generate the database, the software require developer input for proper functionality.

\subsubsection{Terminology and Definitions}
I added and further explained the terminologies I used in the document, thanks to the peer reviewers who gave me advice on what was not clear.

\subsubsection{Goal statements}
As the feedback from SRS presentation and from Dr. Smith, I refine the goals to one goal, which is predicting the chloride exposure for bridges in Ontario in the past and future and allow user to input coordinate and return the prediction for that location.

\subsubsection{Functional Requirements}
From the feedback of Dr. Smith, I condensed the five requirements to three, removing the ones that should belong to VnV plan and reflected it in later documents.

\subsubsection{Other Typos}
I also fixed the many typos in SRS, a version of the \href{https://github.com/CynthiaLiu0805/BridgeCorrosion/blob/8e7dd3f2ec9c6140d6d36c7debbd7f42d1aaf747/docs/SRS/SRS.pdf}{origin SRS} is here for comparison with the current version.

\subsection{Design and Design Documentation}
After getting feedback from peer review and Dr. Smith, I implemented significant changes to the \href{https://github.com/CynthiaLiu0805/BridgeCorrosion/tree/main/docs/Design}{design documentation}, primarily focusing on enhancing its level of detail. In the \href{https://github.com/CynthiaLiu0805/BridgeCorrosion/tree/20b1a6d07956f6f319200b020e6b25e1b419a75f/docs/Design}{previous version} of my design document, I omitted the detail formulas for the calculation because I thought they could be found in SRS, but that is not enough for someone who is only reading the design document. So I include those and make it more detail by separating the calculation process by steps. Also, I rename some modules to make the name end with T to indicate it is an abstract data type.

\subsubsection{Unlikely Changes}
There was some ambiguity for UC3, which is about visualization. I rephrase the sentence so that it simply mean ``there would be visualization'' rather than talking about the type of visualization.
\subsubsection{Modules}
In the first version of my software, I put the calculation process within one module, but then I realize that is not detailed enough and it will be hard for information hiding, as each step of the calculation has its own secret, which is the different formula. So I separate the calculation by steps, with one step aiming to solve a single equation. This is also good for maintainability, if the formula for one step changes, only the corresponding module need to be changed. The Use Hierarchy changes correspondingly as well, the comparsion could be found in Section \ref{appendix}.

\subsection{VnV Plan and Report}
The VnV Plan changed a lot as well because there was a coding language change between my \href{https://github.com/CynthiaLiu0805/BridgeCorrosion/blob/3a78fea11558b0c5a886bfcc70a0b7bec5653821/docs/VnVPlan/VnVPlan.pdf}{initial drafting} and the \href{https://github.com/CynthiaLiu0805/BridgeCorrosion/blob/main/docs/VnVPlan/VnVPlan.pdf}{completion} of MG and MIS.

\subsubsection{Automated Testing and Verification Tools}
This is the section that includes the coding language. In the beginning, I planned to use Matlab to generate the database and Python to make the software that users are interacting with, but after getting feedback from Dr. Smith I changed it to do everything in Python. So this section changes correspondingly. 

\subsubsection{System Test Description}
One thing I did for consistency is that I updated modified the values in table to be the ones I actually used in test cases. I also moved the origin ``Intermediate Tests - test model calculation'' to the unit test section as suggested.

\subsubsection{Unit Test Description}
I finished this section after MG and MIS, all the test cases could be found in the \href{https://github.com/CynthiaLiu0805/BridgeCorrosion/tree/main/tests}{test folder}, and in this section I also provide reference link to the files. The detail explanation for each test case is included in the code comments, with this section acts as an overview of the decisions.


\section{Design Iteration (LO11)}

\plt{Explain how you arrived at your final design and implementation.  How did
the design evolve from the first version to the final version?} 

\section{Design Decisions (LO12)}

\plt{Reflect and justify your design decisions.  How did limitations,
 assumptions, and constraints influence your decisions?}

\section{Economic Considerations (LO23)}

\plt{Is there a market for your product? What would be involved in marketing your 
product? What is your estimate of the cost to produce a version that you could 
sell?  What would you charge for your product?  How many units would you have to 
sell to make money? If your product isn't something that would be sold, like an 
open source project, how would you go about attracting users?  How many potential 
users currently exist?}

\section{Reflection on Project Management (LO24)}

\plt{This question focuses on processes and tools used for project management.}

\subsection{How Does Your Project Management Compare to Your Development Plan}

\plt{Did you follow your Development plan, with respect to the team meeting plan, 
team communication plan, team member roles and workflow plan.  Did you use the 
technology you planned on using?}

\subsection{What Went Well?}

\plt{What went well for your project management in terms of processes and 
technology?}

\subsection{What Went Wrong?}

\plt{What went wrong in terms of processes and technology?}

\subsection{What Would you Do Differently Next Time?}

\plt{What will you do differently for your next project?}

\section{Appendix}\label{appendix}

\begin{figure}
\begin{minipage}[c]{0.45\linewidth}
\includegraphics[width=\linewidth]{UsesHierarchy\_before.png}
\caption{UsesHierarchy before}
\end{minipage}
\hfill
\begin{minipage}[c]{0.45\linewidth}
\includegraphics[width=\linewidth]{UsesHierarchy\_after.png}
\caption{UsesHierarchy after}
\end{minipage}%
\end{figure}

\end{document}