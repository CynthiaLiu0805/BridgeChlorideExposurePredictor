\documentclass{article}

\usepackage{tabularx}
\usepackage{booktabs}

\title{Problem Statement and Goals\\A Mathematical Model for Bridge Corrosion}

\author{Cynthia Liu}

\date{Jan 14}

%% Comments

\usepackage{color}

\newif\ifcomments\commentstrue %displays comments
%\newif\ifcomments\commentsfalse %so that comments do not display

\ifcomments
\newcommand{\authornote}[3]{\textcolor{#1}{[#3 ---#2]}}
\newcommand{\todo}[1]{\textcolor{red}{[TODO: #1]}}
\else
\newcommand{\authornote}[3]{}
\newcommand{\todo}[1]{}
\fi

\newcommand{\wss}[1]{\authornote{blue}{SS}{#1}} 
\newcommand{\plt}[1]{\authornote{magenta}{TPLT}{#1}} %For explanation of the template
\newcommand{\an}[1]{\authornote{cyan}{Author}{#1}}

%% Common Parts

\newcommand{\progname}{ProgName} % PUT YOUR PROGRAM NAME HERE
\newcommand{\authname}{Team \#, Team Name
\\ Student 1 name
\\ Student 2 name
\\ Student 3 name
\\ Student 4 name} % AUTHOR NAMES                  

\usepackage{hyperref}
    \hypersetup{colorlinks=true, linkcolor=blue, citecolor=blue, filecolor=blue,
                urlcolor=blue, unicode=false}
    \urlstyle{same}
                                


\begin{document}

\maketitle

\begin{table}[hp]
\caption{Revision History} \label{TblRevisionHistory}
\begin{tabularx}{\textwidth}{llX}
\toprule
\textbf{Date} & \textbf{Developer(s)} & \textbf{Change}\\
\midrule
Jan 14 & Cynthia Liu & initial document\\
... & ... & ...\\
\bottomrule
\end{tabularx}
\end{table}

\section{Problem Statement}

\subsection{Problem}
In Ontario, most of the bridges in highway features reinforced concrete(RC) decks, however, the external factors such as climate change and transport conditions are likely to affect their corrosion, introducing uncertainties about their long-term durability. In this case, it is important to investigate how it might impact corrosion-induced damage for RC bridges, particularly the areas comprising most highway bridges. Understanding these potential effects will enable informed decision-making in terms of maintenance, design adjustments, and long-term planning to ensure the resilience and durability of the region's vital infrastructure.

\subsection{Inputs and Outputs}
The input of the software is information of a location such as coordinate. The output is a predictive model forecasting the likelihood of corrosion in bridges situated at the provided coordinates over the next five or ten years.

\subsection{Stakeholders}
\begin{itemize}
\item Government: Government entities want to see the accurate data and use it as a reference to plan the road construction of Ontario. For example, they might want to adjust the thickness of road in certain area based on the data coming from this model.
\item Researcher: Professionals within the civil engineering field who are concerned with the impacts of climate change on corrosion damage. 
\item Developer: Individuals involved in the development, maintenance, and potential future enhancements of the predictive model. This group may include software developers or engineers tasked with inheriting, sustaining, and advancing the model beyond the initial research phase.
\end{itemize}


\subsection{Environment}
This software is accessible through any standard web browser and any operating system. The browser should be one of the several recent versions of Google Chrome, Firefox, MS Edge and Safari. The operating system include Windows 7+ and Mac OS X 10.7+.

\section{Goals}
\begin{itemize}
\item This software will provide a predictive model indicating the potential trends of corrosion based on location information, the data and result should be accurate and reliable.
\item This software will be easy to use and be accessible to users with all backgrounds. 
\item This software will be able to handle the exception or produce error messages when the input is out of constraint.
\end{itemize}

\section{Stretch Goals}
\begin{itemize}
\item This software will visualize the output to enhance its public appeal and providing a clearer depiction of trends.
\item This software will exhibit adaptability for deployment in diverse regions contingent upon the availability of pertinent databases.
\end{itemize}



\end{document}