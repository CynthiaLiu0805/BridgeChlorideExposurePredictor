\documentclass[12pt, titlepage]{article}

\usepackage{booktabs}
\usepackage{tabularx}
\usepackage{hyperref}
\hypersetup{
    colorlinks,
    citecolor=black,
    filecolor=black,
    linkcolor=red,
    urlcolor=blue
}
\usepackage[round]{natbib}

\input{../Comments}
\input{../Common}

\begin{document}

\title{Verification and Validation Report: Bridge Corrosion} 
\author{Cynthia Liu}
\date{\today}
	
\maketitle

\pagenumbering{roman}

\section{Revision History}

\begin{tabularx}{\textwidth}{p{3cm}p{2cm}X}
\toprule {\bf Date} & {\bf Version} & {\bf Notes}\\
\midrule
Apr. 14 2024 & 1.0 & Initial release\\
\bottomrule
\end{tabularx}

~\newpage

\section{Symbols, Abbreviations and Acronyms}
See SRS Documentation at \href{https://github.com/CynthiaLiu0805/BridgeCorrosion/blob/main/docs/SRS/SRS.pdf}{SRS}.
The symbols, abbreviations and acronyms used in \href{https://github.com/CynthiaLiu0805/BridgeCorrosion/blob/main/docs/Design/SoftArchitecture/MG.pdf}{MG}, \href{https://github.com/CynthiaLiu0805/BridgeCorrosion/blob/main/docs/Design/SoftDetailedDes/MIS.pdf}{MIS}, \href{https://github.com/CynthiaLiu0805/BridgeCorrosion/blob/main/docs/VnVPlan/VnVPlan.pdf}{VnVPlan} also applied here.


%\wss{symbols, abbreviations or acronyms -- you can reference the SRS tables if needed}

\newpage

\tableofcontents

\listoftables %if appropriate

%\listoffigures %if appropriate

\newpage

\pagenumbering{arabic}

This document includes the results of \href{https://github.com/CynthiaLiu0805/BridgeCorrosion/blob/main/docs/VnVPlan/VnVPlan.pdf}{VnVPlan}. 

\section{Functional Requirements Evaluation}\label{FRE}
This section covers the tests of the functional requirements.
\subsection{Input tests - Coordinate}
This test classifies the valid and invalid input from user. It is covered by unit testing in \href{https://github.com/CynthiaLiu0805/BridgeCorrosion/blob/main/src/app/test_input_check.py}{test\_input\_check.py}. All the test cases passed successfully. 


\subsection{Input tests - Models}
This test check if there are any missing values from the input climate and traffic model. It is achieved by unit testing in  \href{https://github.com/CynthiaLiu0805/BridgeCorrosion/blob/main/src/database/test_model_check.py}{test\_model\_input.py}. All the test cases passed successfully. 

\subsection{Output tests - Chloride exposure}
This test checks if the output is reliable by comparing the actual result with the result from Matlab, which is also implemented by the author as a second way to ensure the accuracy. It is achieved by unit testing in  \href{https://github.com/CynthiaLiu0805/BridgeCorrosion/blob/main/src/database/test_calculation.py}{test\_calculation.py}. All the test cases passed successfully. 

\section{Nonfunctional Requirements Evaluation}
This section covers the tests of the non-functional requirements. 
\subsection{Reliability}
The reliability is tested through section \ref{FRE} and section \ref{UT}.

\subsection{Usability}
The author did a demonstration during the final presentation and shared it with some friends. Feedback from participants included:
\begin{itemize}
\item The interactive buttons for output graphs were positioned too closely to the map.
\item After obtaining results for one location, the map reverted to its initial state. This resulted in delays for users trying to compare two nearby locations.
\item The Key and Value for the result list is confusing, participants suggested using more descriptive text to improve clarity.
\item There could be the 25 $\times$ 25 grid shown on the map to visualize the distance.
\end{itemize}
Due to time constraints, these comments were not addressed immediately. However, plans are in place to address them in the future to enhance the usability and clarity of the demonstration.

\subsection{Maintainability}
The author did a code review with friends from different background in McMaster and the most common comments are as follows:
\begin{itemize}
\item The relationship between modules are not clear without looking at module interface specification
\item Functions lack comments or explanations regarding their purpose, which can make understanding the code more challenging.
\end{itemize}
The primary approach I used to enhance maintainability was to add comments and explanations in the code, ensuring easier comprehension for future developers.


\subsection{Portability}
The author used Github Actions to test if the software could be installed in multiple machines and environments. Some manual testing is done by asking fellows to run the project following the \href{https://github.com/CynthiaLiu0805/BridgeCorrosion/blob/main/README.md}{Readme.md}, most of the process went smoothly, thought there are some problems:
\begin{itemize}
\item When installing make, user need to restart the Terminal or PowerShell to make it work. This is added to the readme.
\item To run the python files, ``\textit{python filename.py}" works on most computers while some need to be ``\textit{python3 filename.py}". In the makefile it is set to python by default, but the alternative way to run section is added to readme for cases with python3.
\end{itemize}

\section{Unit Testing}\label{UT}
This section shows the result of the other unit testing for this software. To re-generate the result, you can run ``\textit{make test}'' or ``\textit{pytest src/database/testfile.py}'' in the root folder of this repo, replacing \textit{testfile} with the actual file names. 

\subsection{test\_calculation}
All the test cases pass successfully thought there are some warnings. Check the \href{https://github.com/CynthiaLiu0805/BridgeCorrosion/blob/main/tests/test\_calculation.log}{log} for detail.


\subsection{test\_model\_input}
All the test cases pass successfully. Check the \href{https://github.com/CynthiaLiu0805/BridgeCorrosion/blob/main/tests/test\_model\_input.log}{log} for detail.


\subsection{test\_input\_check}
All the test cases pass successfully. Check the \href{https://github.com/CynthiaLiu0805/BridgeCorrosion/blob/main/tests/test\_input\_check.log}{log} for detail.

\subsection{test\_search}
All the test cases pass successfully. Check the \href{https://github.com/CynthiaLiu0805/BridgeCorrosion/blob/main/tests/test\_search.log}{log} for detail.
\subsection{test\_visualization}
All the test cases pass successfully. Check the \href{https://github.com/CynthiaLiu0805/BridgeCorrosion/blob/main/tests/test\_visualization.log}{log} for detail.

\section{Changes Due to Testing}
Many minor bugs are fixed and more explanation are added for maintainability and portability. One important change is regarding precision: in the beginning the output results were rounded to two decimal places. However, after careful consideration, it seemed like that this level of precision might not be sufficient. This is to be discussed with Dr. Yang and Mingsai, but I dicide not to round it for now. Meanwhile, in the visualization, I still keep it as two decimal for readability when users putting mouses on the points on graph.  
%\wss{This section should highlight how feedback from the users and from the supervisor (when one exists) shaped the final product.  In particular the feedback from the Rev 0 demo to the supervisor (or to potential users) should be highlighted.}

\section{Automated Testing}
Continuous Integration (CI) is used for automated testing. The test cases using pytest are set up in \href{https://github.com/CynthiaLiu0805/BridgeCorrosion/actions}{GitHub Actions}. Whenever there is a new push, CI will run through all test cases to check if everything's working as expected.

\section{Trace to Requirements}

\begin{table}[h]
\centering
\begin{tabular}{|c|c|c|c|c|c|c|c|c|c|c|}
\hline
	& R1 & R2 & R3  & NFR1 & NFR2 & NFR3 & NFR4  \\
\hline
\href{https://github.com/CynthiaLiu0805/BridgeCorrosion/blob/main/src/database/test_calculation.py}{t\_calculation} & & X & X & X & & &  \\ \hline
\href{https://github.com/CynthiaLiu0805/BridgeCorrosion/blob/main/src/database/test_model_check.py}{t\_model\_input.py} & X & & & & & &  \\ \hline
\href{https://github.com/CynthiaLiu0805/BridgeCorrosion/blob/main/src/app/test_input_check.py}{t\_input\_check.py}        & X & & & & & &  \\ \hline
\href{https://github.com/CynthiaLiu0805/BridgeCorrosion/blob/main/src/app/test_search.py}{t\_search.py}  & & X & & & & &  \\ \hline
\href{https://github.com/CynthiaLiu0805/BridgeCorrosion/blob/main/src/app/test_visualization.py}{t\_visualization} & & & & & X & &  \\ \hline
Test Reliability       & & & & X & & &  \\ \hline
Test Usability       & & & & & X & &  \\ \hline
Test Maintainability        & & & & & & X &  \\ \hline
Test Portability        & & & & & & & X  \\ \hline

\end{tabular}
\caption{Traceability Between Test Cases and Requirements}
\label{Table:test_requirements}
\end{table}


\section{Trace to Modules}		
\begin{table}[h]
\centering

\begin{tabular}{|c|p{4mm}|p{4mm}|p{4mm}|p{4mm}|p{4mm}|p{4mm}|p{4mm}|p{4mm}|p{6mm}|p{6mm}|p{6mm}|p{6mm}|p{6mm}|}
\hline
	& M2 & M3 & M4  & M5 & M6 & M7 & M8 & M9 & M10 & M11 & M12 & M13 & M14 \\
\hline

\href{https://github.com/CynthiaLiu0805/BridgeCorrosion/blob/main/src/database/test_calculation.py}{t\_calculation}    & &  &  &  & X & X & X & X & X & X & X & X & X \\ \hline
\href{https://github.com/CynthiaLiu0805/BridgeCorrosion/blob/main/src/database/test_model_check.py}{t\_model\_input} & &  &  &  & X &  & &  &  &  &  &  & \\ \hline
\href{https://github.com/CynthiaLiu0805/BridgeCorrosion/blob/main/src/app/test_input_check.py}{t\_input\_check} & X & X &  &  &  &  & &  &  &  &  &  & \\ \hline
\href{https://github.com/CynthiaLiu0805/BridgeCorrosion/blob/main/src/app/test_search.py}{t\_search.py} & & X & X &  &  &  & &  &  &  &  &  & \\ \hline
\href{https://github.com/CynthiaLiu0805/BridgeCorrosion/blob/main/src/app/test_visualization.py}{t\_visualization} & &  &  & X &  &  & &  &  &  &  &  & \\ \hline

\end{tabular}
\caption{Traceability Between Test Cases and Modules}
\label{Table:test_modules}
\end{table}

\bibliographystyle{plainnat}
\bibliography{../../refs/References}

\newpage{}

\end{document}