\documentclass[12pt, titlepage]{article}

\usepackage{booktabs}
\usepackage{tabularx}
\usepackage{hyperref}
\hypersetup{
    colorlinks,
    citecolor=black,
    filecolor=black,
    linkcolor=red,
    urlcolor=blue
}
\usepackage[round]{natbib}

%% Comments

\usepackage{color}

\newif\ifcomments\commentstrue %displays comments
%\newif\ifcomments\commentsfalse %so that comments do not display

\ifcomments
\newcommand{\authornote}[3]{\textcolor{#1}{[#3 ---#2]}}
\newcommand{\todo}[1]{\textcolor{red}{[TODO: #1]}}
\else
\newcommand{\authornote}[3]{}
\newcommand{\todo}[1]{}
\fi

\newcommand{\wss}[1]{\authornote{blue}{SS}{#1}} 
\newcommand{\plt}[1]{\authornote{magenta}{TPLT}{#1}} %For explanation of the template
\newcommand{\an}[1]{\authornote{cyan}{Author}{#1}}

%% Common Parts

\newcommand{\progname}{ProgName} % PUT YOUR PROGRAM NAME HERE
\newcommand{\authname}{Team \#, Team Name
\\ Student 1 name
\\ Student 2 name
\\ Student 3 name
\\ Student 4 name} % AUTHOR NAMES                  

\usepackage{hyperref}
    \hypersetup{colorlinks=true, linkcolor=blue, citecolor=blue, filecolor=blue,
                urlcolor=blue, unicode=false}
    \urlstyle{same}
                                


\begin{document}

\title{Verification and Validation Report: Bridge Corrosion} 
\author{Cynthia Liu}
\date{\today}
	
\maketitle

\pagenumbering{roman}

\section{Revision History}

\begin{tabularx}{\textwidth}{p{3cm}p{2cm}X}
\toprule {\bf Date} & {\bf Version} & {\bf Notes}\\
\midrule
Apr. 14 2024 & 1.0 & Initial release\\
\bottomrule
\end{tabularx}

~\newpage

\section{Symbols, Abbreviations and Acronyms}
See SRS Documentation at \href{https://github.com/CynthiaLiu0805/BridgeCorrosion/blob/main/docs/SRS/SRS.pdf}{SRS}.


%\wss{symbols, abbreviations or acronyms -- you can reference the SRS tables if needed}

\newpage

\tableofcontents

\listoftables %if appropriate

\listoffigures %if appropriate

\newpage

\pagenumbering{arabic}

This document includes the results of \href{https://github.com/CynthiaLiu0805/BridgeCorrosion/blob/main/docs/VnVPlan/VnVPlan.pdf}{VnVPlan}. To re-generate the result, you can run ``make test'' in the root folder of this repo. 

\section{Functional Requirements Evaluation}
This section covers the tests of the functional requirements.
\subsection{Input tests - Coordinate}
This test classifies the valid and invalid input from user. It is covered by unit testing in \href{https://github.com/CynthiaLiu0805/BridgeCorrosion/blob/main/src/app/test_input_check.py}{test\_input\_check.py}. All the test cases passed successfully. 


\subsection{Input tests - Models}
This test check if there are any missing values from the input climate and traffic model. It is achieved by unit testing in  \href{https://github.com/CynthiaLiu0805/BridgeCorrosion/blob/main/src/database/test_model_check.py}{test\_model\_input.py}. All the test cases passed successfully. 

\section{Nonfunctional Requirements Evaluation}
This section covers the tests of the non-functional requirements.

\subsection{Usability}
		
\subsection{Performance}



\section{Unit Testing}

\section{Changes Due to Testing}

\wss{This section should highlight how feedback from the users and from 
the supervisor (when one exists) shaped the final product.  In particular 
the feedback from the Rev 0 demo to the supervisor (or to potential users) 
should be highlighted.}

\section{Automated Testing}
		
\section{Trace to Requirements}
		
\begin{table}[h]
\centering
\begin{tabular}{|c|c|c|c|c|c|c|c|c|c|c|}
\hline
	& R1 & R2 & R3  & NFR1 & NFR2 & NFR3 & NFR4  \\
\hline
\href{https://github.com/CynthiaLiu0805/BridgeCorrosion/blob/main/src/app/test_input_check.py}{t\_input\_check.py}        & X & & & & & &  \\ \hline
\href{https://github.com/CynthiaLiu0805/BridgeCorrosion/blob/main/src/database/test_model_check.py}{t\_model\_input.py} & X & & & & & &  \\ \hline
Test \ref{t_output}        & & X & X & & & &  \\ \hline
Test \ref{t_reliability}        & & & & X & & &  \\ \hline
Test \ref{t_usability}        & & & & & X & &  \\ \hline
Test \ref{t_maintainability}        & & & & & & X &  \\ \hline
Test \ref{t_portability}        & & & & & & & X  \\ \hline

\end{tabular}
\caption{Traceability Between Test Cases and Requirements}
\label{Table:test_requirements}
\end{table}


\section{Trace to Modules}		
\begin{table}[h]
\centering
\begin{tabular}{|c|c|c|c|c|c|c|c|c|c|c|c|c|c|}
\hline
	& M2 & M3 & M4  & M5 & M6 & M7 & M8 & M9 & M10 & M11 & M12 & M13 & M14 \\
\hline

\href{https://github.com/CynthiaLiu0805/BridgeCorrosion/blob/main/src/database/test_calculation.py}{t\_calculation}    & &  &  &  & X & X & X & X & X & X & X & X & X \\ \hline
\href{https://github.com/CynthiaLiu0805/BridgeCorrosion/blob/main/src/database/test_model_check.py}{t\_model\_input} & &  &  &  & X &  & &  &  &  &  &  & \\ \hline
\href{https://github.com/CynthiaLiu0805/BridgeCorrosion/blob/main/src/app/test_input_check.py}{t\_input\_check} & X & X &  &  &  &  & &  &  &  &  &  & \\ \hline
\href{https://github.com/CynthiaLiu0805/BridgeCorrosion/blob/main/src/app/test_search.py}{t\_search.py} & & X & X &  &  &  & &  &  &  &  &  & \\ \hline
\href{https://github.com/CynthiaLiu0805/BridgeCorrosion/blob/main/src/app/test_visualization.py}{t\_visualization} & &  &  & X &  &  & &  &  &  &  &  & \\ \hline

\end{tabular}
\caption{Traceability Between Test Cases and Modules}
\label{Table:test_modules}
\end{table}
\section{Code Coverage Metrics}

\bibliographystyle{plainnat}
\bibliography{../../refs/References}

\newpage{}

\end{document}