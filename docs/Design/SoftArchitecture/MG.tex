\documentclass[12pt, titlepage]{article}

\usepackage{fullpage}
\usepackage[round]{natbib}
\usepackage{multirow}
\usepackage{booktabs}
\usepackage{tabularx}
\usepackage{graphicx}
\usepackage{float}
\usepackage{hyperref}
\hypersetup{
    colorlinks,
    citecolor=blue,
    filecolor=black,
    linkcolor=red,
    urlcolor=blue
}

\input{../../Comments}
\input{../../Common}

\newcounter{acnum}
\newcommand{\actheacnum}{AC\theacnum}
\newcommand{\acref}[1]{AC\ref{#1}}

\newcounter{ucnum}
\newcommand{\uctheucnum}{UC\theucnum}
\newcommand{\uref}[1]{UC\ref{#1}}

\newcounter{mnum}
\newcommand{\mthemnum}{M\themnum}
\newcommand{\mref}[1]{M\ref{#1}}


\newcounter{reqnum} %Requirement Number
\newcommand{\rthereqnum}{R\thereqnum}
\newcommand{\rref}[1]{R\ref{#1}}
\newcounter{nfrnum} %NFR Number
\newcommand{\rthenfrnum}{NFR\thenfrnum}
\newcommand{\nfrref}[1]{NFR\ref{#1}}

\begin{document}

\title{Module Guide for Bridge Corrosion} 
\author{Cynthia Liu}
\date{\today}

\maketitle

\pagenumbering{roman}

\section{Revision History}

\begin{tabularx}{\textwidth}{p{3cm}p{2cm}X}
\toprule {\bf Date} & {\bf Version} & {\bf Notes}\\
\midrule
March 6, 2024 & 0.0 & Initial release\\
March 18, 2024 & 0.1 & Edit according to feedback from peer review \\
April 5, 2024 & 1 & Edit according to feedback from Dr. Smith\\
\bottomrule
\end{tabularx}

\newpage

\section{Reference Material}

This section records information for easy reference.

\subsection{Abbreviations and Acronyms}

\renewcommand{\arraystretch}{1.2}
\begin{tabular}{l l} 
  \toprule		
  \textbf{symbol} & \textbf{description}\\
  \midrule 
  AC & Anticipated Change\\
  BC & Bridge Corrosion\\
  DAG & Directed Acyclic Graph \\
  GUI & Graphical User Interface \\
  M & Module \\
  MG & Module Guide \\
  OS & Operating System \\
  R & Requirement\\
  SAS & Sprayed and Splashed \\
  SC & Scientific Computing \\
  SRS & Software Requirements Specification\\
  UC & Unlikely Change \\
  \bottomrule
\end{tabular}\\

\newpage

\tableofcontents

\listoftables

\listoffigures

\newpage

\pagenumbering{arabic}

\section{Introduction}

Decomposing a system into modules is a commonly accepted approach to developing
software.  A module is a work assignment for a programmer or programming
team.  We advocate a decomposition
based on the principle of information hiding.  This
principle supports design for change, because the ``secrets'' that each module
hides represent likely future changes.  Design for change is valuable in SC,
where modifications are frequent, especially during initial development as the
solution space is explored.  

Our design follows the rules layed out by, as follows:
\begin{itemize}
\item System details that are likely to change independently should be the
  secrets of separate modules.
\item Each data structure is implemented in only one module.
\item Any other program that requires information stored in a module's data
  structures must obtain it by calling access programs belonging to that module.
\end{itemize}

After completing the first stage of the design, the Software Requirements
Specification (SRS), the Module Guide (MG) is developed. The MG
specifies the modular structure of the system and is intended to allow both
designers and maintainers to easily identify the parts of the software.  The
potential readers of this document are as follows:

\begin{itemize}
\item New project members: This document can be a guide for a new project member
  to easily understand the overall structure and quickly find the
  relevant modules they are searching for.
\item Maintainers: The hierarchical structure of the module guide improves the
  maintainers' understanding when they need to make changes to the system. It is
  important for a maintainer to update the relevant sections of the document
  after changes have been made.
\item Designers: Once the module guide has been written, it can be used to
  check for consistency, feasibility, and flexibility. Designers can verify the
  system in various ways, such as consistency among modules, feasibility of the
  decomposition, and flexibility of the design.
\end{itemize}

The rest of the document is organized as follows. Section
\ref{SecChange} lists the anticipated and unlikely changes of the software
requirements. Section \ref{SecMH} summarizes the module decomposition that
was constructed according to the likely changes. Section \ref{SecConnection}
specifies the connections between the software requirements and the
modules. Section \ref{SecMD} gives a detailed description of the
modules. Section \ref{SecTM} includes two traceability matrices. One checks
the completeness of the design against the requirements provided in the SRS. The
other shows the relation between anticipated changes and the modules. Section
\ref{SecUse} describes the use relation between modules.

\section{Anticipated and Unlikely Changes} \label{SecChange}

This section lists possible changes to the system. According to the likeliness
of the change, the possible changes are classified into two
categories. Anticipated changes are listed in Section \ref{SecAchange}, and
unlikely changes are listed in Section \ref{SecUchange}.

\subsection{Anticipated Changes} \label{SecAchange}

Anticipated changes are the source of the information that is to be hidden
inside the modules. Ideally, changing one of the anticipated changes will only
require changing the one module that hides the associated decision. The approach
adapted here is called design for change.

\begin{description}
\item[\refstepcounter{acnum} \actheacnum \label{acMoreCoordinate}:] Coordinate outside Ontario might be considered if the project becomes a larger scale. This is relevant to R1 in \href{https://github.com/CynthiaLiu0805/BridgeCorrosion/blob/main/docs/SRS/SRS.pdf}{SRS}.
\item[\refstepcounter{acnum} \actheacnum \label{acAlgorithm}:] The algorithm for calculating chloride exposure might change if the theory proceeds.
\item[\refstepcounter{acnum} \actheacnum \label{acOutputFile}:] The output could have another format which is saving the search result to file.

\end{description}

\subsection{Unlikely Changes} \label{SecUchange}

The module design should be as general as possible. However, a general system is
more complex. Sometimes this complexity is not necessary. Fixing some design
decisions at the system architecture stage can simplify the software design. If
these decision should later need to be changed, then many parts of the design
will potentially need to be modified. Hence, it is not intended that these
decisions will be changed.

\begin{description}
\item[\refstepcounter{ucnum} \uctheucnum \label{ucDevInput}:] The traffic data model and climate data model are inputed from developer and will be hidden to user.
\item[\refstepcounter{ucnum} \uctheucnum \label{ucUserInput}:] The user input is only the coordinate.
\item [\refstepcounter{ucnum} \uctheucnum \label{ucOutput}:] There will be visualization for the output.
\item [\refstepcounter{ucnum} \uctheucnum \label{ucDatabase}:] The database storing chloride exposure data will always be external.
\end{description}

\section{Module Hierarchy} \label{SecMH}

This section provides an overview of the module design. Modules are summarized
in a hierarchy decomposed by secrets in Table \ref{TblMH}. The modules listed
below, which are leaves in the hierarchy tree, are the modules that will
actually be implemented.\\
The whole project could be divided into two parts. \mref{mHardware} to \mref{mOutput} are the software that user will actually interact with, and \mref{mDataModel} to \mref{mCalculation_result} are the modules that generate the chloride exposure database that the software depends on. The purpose for divding the process of generating the database is because each module hides a secret of the calculation formula, so if one of the formula changes only the corresponding module need to be changed. \mref{mFileIO} and \mref{mPlot} are the built in modules by Python that is used during the implementation.

\begin{description}
\item [\refstepcounter{mnum} \mthemnum \label{mHardware}:] Hardware Hiding Module
\item [\refstepcounter{mnum} \mthemnum \label{mInput}:] Input Module
\item [\refstepcounter{mnum} \mthemnum \label{mControl}:] Control Module
\item [\refstepcounter{mnum} \mthemnum \label{mSearch}:] Data Searching Module
\item [\refstepcounter{mnum} \mthemnum \label{mOutput}:] Output Visualization Module
\item [\refstepcounter{mnum} \mthemnum \label{mDataModel}:] Data Model Reading Module 
\item [\refstepcounter{mnum} \mthemnum \label{mConstant}:] Constant Module 
\item [\refstepcounter{mnum} \mthemnum \label{mCalculation_step1}:] Deicing Salts Calculation Module 
\item [\refstepcounter{mnum} \mthemnum \label{mCalculation_step2}:] Melted Water Thickness Calculation Module 
\item [\refstepcounter{mnum} \mthemnum \label{mCalculation_step3}:] Single Water SAS Calculation Module 
\item [\refstepcounter{mnum} \mthemnum \label{mCalculation_step4}:] Single Chloride Ions SAS Calculation Module 
\item [\refstepcounter{mnum} \mthemnum \label{mCalculation_step5}:] All Chloride Ions SAS Calculation Module 
\item [\refstepcounter{mnum} \mthemnum \label{mCalculation_step6}:] Chloride Deposition Calculation Module 
\item [\refstepcounter{mnum} \mthemnum \label{mCalculation_result}:] Chloride Exposure Database Generation Module
\item [\refstepcounter{mnum} \mthemnum \label{mFileIO}:] File I/O Module
\item [\refstepcounter{mnum} \mthemnum \label{mPlot}:] Plotting Module
\end{description}


\begin{table}[h!]
\centering
\begin{tabular}{p{0.3\textwidth} p{0.6\textwidth}}
\toprule
\textbf{Level 1} & \textbf{Level 2}\\
\midrule

{Hardware-Hiding Module} & ~ \\
\midrule

\multirow{14}{0.3\textwidth}{Behaviour-Hiding Module} & Input Module\\
& Control Module\\
& Data Searching Module\\
& Output Visualization Module\\
& Data Model Reading Module \\
& Constant Module \\
& Deicing Salts Calculation Module \\
& Melted Water Thickness Calculation Module \\
& Single Water SAS Calculation Module \\
& Single Chloride Ions SAS Calculation Module \\
& All Chloride Ions SAS Calculation Module \\
& Chloride Deposition Calculation Module \\
& Chloride Exposure Database Generation Module\\
\midrule

\multirow{2}{0.3\textwidth}{Software Decision Module} &  File I/O Module \\
& Plotting Module \\
\bottomrule

\end{tabular}
\caption{Module Hierarchy}
\label{TblMH}
\end{table}

\section{Connection Between Requirements and Design} \label{SecConnection}

The design of the system is intended to satisfy the requirements developed in
the SRS. In this stage, the system is decomposed into modules. The connection
between requirements and modules is listed in Table~\ref{TblRT}. 

%\wss{The intention of this section is to document decisions that are made
 % ``between'' the requirements and the design.  To satisfy some requirements,
 % design decisions need to be made.  Rather than make these decisions implicit,
  %they are explicitly recorded here.  For instance, if a program has security
  %requirements, a specific design decision may be made to satisfy those
 % requirements with a password.}

\section{Module Decomposition} \label{SecMD}

Modules are decomposed according to the principle of ``information hiding''. The \emph{Secrets} field in a module
decomposition is a brief statement of the design decision hidden by the
module. The \emph{Services} field specifies \emph{what} the module will do
without documenting \emph{how} to do it. For each module, a suggestion for the
implementing software is given under the \emph{Implemented By} title. If the
entry is \emph{OS}, this means that the module is provided by the operating
system or by standard programming language libraries.  \emph{\progname{}} means the
module will be implemented by the \progname{} software.

Only the leaf modules in the hierarchy have to be implemented. If a dash
(\emph{--}) is shown, this means that the module is not a leaf and will not have
to be implemented.

\subsection{Hardware Hiding Modules (\mref{mHardware})}

\begin{description}
\item[Secrets:]The data structure and algorithm used to implement the virtual
  hardware.
\item[Services:]Serves as a virtual hardware used by the rest of the
  system. This module provides the interface between the hardware and the
  software. So, the system can use it to display outputs or to accept inputs.
\item[Implemented By:] OS
\end{description}

\subsection{Behaviour-Hiding Module}

\begin{description}
\item[Secrets:]The contents of the required behaviours.
\item[Services:]Includes programs that provide externally visible behaviour of
  the system as specified in the software requirements specification (SRS)
  documents. This module serves as a communication layer between the
  hardware-hiding module and the software decision module. The programs in this
  module will need to change if there are changes in the SRS.
\item[Implemented By:] --
\end{description}

\subsubsection{Input Module (\mref{mInput})}

\begin{description}
\item[Secrets:] The kind of valid and invalid input coordinate data. 
\item[Services:] Get input from the user and verify if the input is within the physical and software constraint, throw warning message if it is not. 
\item[Implemented By:] BC
\item[Type of Module:] Abstract Data Type
\end{description}


\subsubsection{Control Module (\mref{mControl})}
\begin{description}
\item[Secrets:] The algorithm for running the program.
\item[Services:] Provide the main program and the GUI.
\item[Implemented By:] BC
\item[Type of Module:] Library
\end{description}

\subsubsection{Data Searching Module (\mref{mSearch})}
\begin{description}
\item[Secrets:] The database structure.
\item[Services:] Find the data needed in the database by searching algorithm.
\item[Implemented By:] BC
\item[Type of Module:] Abstract Data Type
\end{description}

\subsubsection{Output Visualization Module (\mref{mOutput})}
\begin{description}
\item[Secrets:] The format and structure of the output chloride exposure data.
\item[Services:] Provide the visualization of the resulting chloride exposure trend, using line graph or histogram. 
\item[Implemented By:] BC
\item[Type of Module:] Library
\end{description}


\subsubsection{Data Model Reading Module (\mref{mDataModel})}
\begin{description}
\item[Secrets:] The origin climate model and traffic model that the calculation is based on.
\item[Services:]  Read data from the climate model and traffic model and store them in the data structure needed for further calculation.
\item[Implemented By:] BC
\item[Type of Module:] Abstract Data Type
\end{description}


\subsubsection{Constant Module (\mref{mConstant})}
\begin{description}
\item[Secrets:] The constant used during calculation.
\item[Services:] Store the constant for mantainability purpose.
\item[Implemented By:] BC
\item[Type of Module:] Abstract Data Type
\end{description}

\subsubsection{Deicing Salts Calculation Module (\mref{mCalculation_step1})}
\begin{description}
\item[Secrets:] The formula to determine the quantity of deicing salts applied per day with snowfall.
\item[Services:] Calculate and store the quantity of deicing salts applied per day with snowfall.
\item[Implemented By:] BC
\item[Type of Module:] Abstract Object
\end{description}

\subsubsection{Melted Water Thickness Calculation Module (\mref{mCalculation_step2})}
\begin{description}
\item[Secrets:] The formula to determine the thickness of melted water per day with snow melting.
\item[Services:] Calculate and store the thickness of melted water per day with snow melting.
\item[Implemented By:] BC
\item[Type of Module:] Abstract Object
\end{description}

\subsubsection{Single Water SAS Calculation Module (\mref{mCalculation_step3})}
\begin{description}
\item[Secrets:] The formula to determine water sprayed and splashed by one truck using a Computational Fluid Dynamics (CFD)-based analytical model.
\item[Services:] Calculate and store the water sprayed and splashed by one truck.
\item[Implemented By:] BC
\item[Type of Module:] Abstract Object
\end{description}

\subsubsection{Single Chloride Ions SAS Calculation Module (\mref{mCalculation_step4})}
\begin{description}
\item[Secrets:] The formula to determine chloride ions sprayed and splashed by one truck.
\item[Services:] Calculate and store the chloride ions sprayed and splashed by one truck.
\item[Implemented By:] BC
\item[Type of Module:] Abstract Object
\end{description}

\subsubsection{All Chloride Ions SAS Calculation Module (\mref{mCalculation_step5})}
\begin{description}
\item[Secrets:] The formula to determine chloride ions sprayed and splashed by all vehicles in one winter season.
\item[Services:] Calculate and store the chloride ions sprayed and splashed by all vehicles in one winter season.
\item[Implemented By:] BC
\item[Type of Module:] Abstract Object
\end{description}

\subsubsection{Chloride Deposition Calculation Module (\mref{mCalculation_step6})}
\begin{description}
\item[Secrets:] The formula to determine the deposition of chloride on the surface of the bridge substructure.
\item[Services:] Calculate and store the deposition of chloride on the surface of the bridge substructure.
\item[Implemented By:] BC
\item[Type of Module:] Abstract Object
\end{description}

\subsubsection{Chloride Exposure Database Generation Module (\mref{mCalculation_result})}
\begin{description}
\item[Secrets:] The detail formula for calculating the chloride exposure database.
\item[Services:] Save the chloride exposure data to the database to be used by the user end.
\item[Implemented By:] BC
\item[Type of Module:] Abstract Object
\end{description}


\subsection{Software Decision Module}

\begin{description}
\item[Secrets:] The design decision based on mathematical theorems, physical
  facts, or programming considerations. The secrets of this module are
  \emph{not} described in the SRS.
\item[Services:] Includes data structure and algorithms used in the system that
  do not provide direct interaction with the user. 
  % Changes in these modules are more likely to be motivated by a desire to
  % improve performance than by externally imposed changes.
\item[Implemented By:] --
\end{description}

\subsubsection{File I/O (\mref{mFileIO})}
\begin{description}
\item[Secrets:] The data structure and algorithms for File I/O 
\item[Services:] Provides functions that could read from file and write to new file. 
\item[Implemented By:] Pandas
\item[Type of Module:] Library
\end{description}

\subsubsection{Plotting Result Module (\mref{mPlot})}
\begin{description}
\item[Secrets:] The data structure and algorithms for plotting data graphically. 
\item[Services:] Provides plot function that can plot the results from Data Searching Module, used by Output Visualization module. 
\item[Implemented By:] Plotly 
\item[Type of Module:] Library
\end{description}

\section{Traceability Matrix} \label{SecTM}

This section shows two traceability matrices: between the modules and the
requirements and between the modules and the anticipated changes.

% the table should use mref, the requirements should be named, use something
% like fref
\begin{table}[H]
\centering
\begin{tabular}{p{0.2\textwidth} p{0.6\textwidth}}
\toprule
\textbf{Req.} & \textbf{Modules}\\
\midrule
R1 & \mref{mHardware}, \mref{mInput}, \mref{mControl}\\
R2 & \mref{mControl}, \mref{mSearch}, \mref{mOutput}, \mref{mPlot}\\
R3 & \mref{mOutput}, \mref{mCalculation_result}\\
NFR1 & \mref{mSearch}, \mref{mCalculation_result}\\
NFR2 & \mref{mControl},  \mref{mSearch}, \mref{mPlot}\\
NFR3 & all modules\\
NFR4 & \mref{mControl}\\
NFR5 & all modules\\

\bottomrule
\end{tabular}
\caption{Trace Between Requirements and Modules}
\label{TblRT}
\end{table}

\begin{table}[H]
\centering
\begin{tabular}{p{0.2\textwidth} p{0.6\textwidth}}
\toprule
\textbf{AC} & \textbf{Modules}\\
\midrule
\acref{acMoreCoordinate} & \mref{mInput}\\
\acref{acAlgorithm} & one of \mref{mCalculation_step1}, \mref{mCalculation_step2}, \mref{mCalculation_step3}, \mref{mCalculation_step4}, \mref{mCalculation_step5}, \mref{mCalculation_step6}\\
\acref{acOutputFile} & \mref{mControl} \\
\bottomrule
\end{tabular}
\caption{Trace Between Anticipated Changes and Modules}
\label{TblACUCT}
\end{table}

\section{Use Hierarchy Between Modules} \label{SecUse}

In this section, the uses hierarchy between modules is
provided. A {\em uses} B if
correct execution of B may be necessary for A to complete the task described in
its specification. That is, A {\em uses} B if there exist situations in which
the correct functioning of A depends upon the availability of a correct
implementation of B.  Figure \ref{FigUH} illustrates the use relation between
the modules. It can be seen that the graph is a directed acyclic graph
(DAG). Each level of the hierarchy offers a testable and usable subset of the
system, and modules in the higher level of the hierarchy are essentially simpler
because they use modules from the lower levels.

\begin{figure}[H]
\centering
\includegraphics[width=0.9\textwidth]{UsesHierarchy.png}
\caption{Use hierarchy among modules}
\label{FigUH}
\end{figure}

%\section*{References}

\section{User Interfaces}

\begin{figure}[H]
\centering
\includegraphics[width=0.9\textwidth]{GUI1.png}
\caption{GUI illustration before inputing coordinate}
\label{FigGUI1}
\end{figure}

\begin{figure}[H]
\centering
\includegraphics[width=0.9\textwidth]{GUI2.png}
\caption{GUI illustration after inputing coordinate}
\label{FigGUI1}
\end{figure}

\section{Timeline}
The specific timeline for module implementation is shown in below table:
\begin{table}[H]
\centering
\begin{tabular}{p{0.6\textwidth} p{0.2\textwidth}  p{0.2\textwidth}}
\toprule
 \textbf{Modules} & \textbf{Finish by} & \textbf{Responsible} \\
\midrule
\mref{mDataModel}, \mref{mConstant} & March 2, 2024& Cynthia Liu\\
\mref{mCalculation_step1}, \mref{mCalculation_step2}, \mref{mCalculation_step3}, \mref{mCalculation_step4}, \mref{mCalculation_step5}, \mref{mCalculation_step6} & March 9, 2024& Cynthia Liu\\
\mref{mCalculation_result} & March 11, 2024& Cynthia Liu\\
\mref{mInput} & March 13, 2024 & Cynthia Liu\\
\mref{mSearch} & March 20, 2024 & Cynthia Liu\\
\mref{mOutput}  & March 27, 2024 & Cynthia Liu\\
\mref{mControl} & March 30, 2024 & Cynthia Liu\\

\bottomrule
\end{tabular}
\caption{Timeline}
\end{table}


\section{Appendix}
This section includes a link to \href{https://github.com/CynthiaLiu0805/BridgeCorrosion/blob/main/docs/SRS/SRS.pdf}{SRS} for requirements.


\bibliographystyle {plainnat}
\bibliography{../../../refs/References}

\newpage{}

\end{document}