\documentclass[12pt, titlepage]{article}

\usepackage{amsmath, mathtools}

\usepackage[round]{natbib}
\usepackage{amsfonts}
\usepackage{amssymb}
\usepackage{graphicx}
\usepackage{colortbl}
\usepackage{xr}
\usepackage{hyperref}
\usepackage{longtable}
\usepackage{xfrac}
\usepackage{tabularx}
\usepackage{float}
\usepackage{siunitx}
\usepackage{booktabs}
\usepackage{multirow}
\usepackage[section]{placeins}
\usepackage{caption}
\usepackage{fullpage}
\usepackage{fixltx2e}

\hypersetup{
bookmarks=true,     % show bookmarks bar?
colorlinks=true,       % false: boxed links; true: colored links
linkcolor=red,          % color of internal links (change box color with linkbordercolor)
citecolor=blue,      % color of links to bibliography
filecolor=magenta,  % color of file links
urlcolor=cyan          % color of external links
}

\usepackage{array}

\externaldocument{../../SRS/SRS}

%% Comments

\usepackage{color}

\newif\ifcomments\commentstrue %displays comments
%\newif\ifcomments\commentsfalse %so that comments do not display

\ifcomments
\newcommand{\authornote}[3]{\textcolor{#1}{[#3 ---#2]}}
\newcommand{\todo}[1]{\textcolor{red}{[TODO: #1]}}
\else
\newcommand{\authornote}[3]{}
\newcommand{\todo}[1]{}
\fi

\newcommand{\wss}[1]{\authornote{blue}{SS}{#1}} 
\newcommand{\plt}[1]{\authornote{magenta}{TPLT}{#1}} %For explanation of the template
\newcommand{\an}[1]{\authornote{cyan}{Author}{#1}}

%% Common Parts

\newcommand{\progname}{ProgName} % PUT YOUR PROGRAM NAME HERE
\newcommand{\authname}{Team \#, Team Name
\\ Student 1 name
\\ Student 2 name
\\ Student 3 name
\\ Student 4 name} % AUTHOR NAMES                  

\usepackage{hyperref}
    \hypersetup{colorlinks=true, linkcolor=blue, citecolor=blue, filecolor=blue,
                urlcolor=blue, unicode=false}
    \urlstyle{same}
                                


\begin{document}

\title{Module Interface Specification for Bridge Corrosion}

\author{Cynthia Liu}

\date{\today}

\maketitle

\pagenumbering{roman}

\section{Revision History}

\begin{tabularx}{\textwidth}{p{3cm}p{2cm}X}
\toprule {\bf Date} & {\bf Version} & {\bf Notes}\\
\midrule
Mar. 8, 2024 & 0 & Initial Release\\
March 18, 2024 & 0.1 & Edit according to feedback from peer review \\
April 5, 2024 & 1 & Edit according to feedback from Dr. Smith\\
July 3, 2024 & 1.1 & Add modules for chloride on deck\\
July 24, 2024 & 1.2 & Match the function name to the actual one \\
\bottomrule
\end{tabularx}

~\newpage

\section{Symbols, Abbreviations and Acronyms}

See SRS Documentation at \href{https://github.com/CynthiaLiu0805/BridgeCorrosion/blob/main/docs/SRS/SRS.pdf}{SRS}.

\wss{Also add any additional symbols, abbreviations or acronyms}

\newpage

\tableofcontents

\newpage

\pagenumbering{arabic}

\section{Introduction}

The following document details the Module Interface Specifications for Bridge Corrosion which investigate how climate, traffic might impact corrosion-induced
damage for reinforced concrete bridges by influencing the chloride exposure.
%\wss{Fill in your project name and description}

Complementary documents include the System Requirement Specifications
and Module Guide.  The full documentation and implementation can be
found at \href{https://github.com/CynthiaLiu0805/BridgeCorrosion}{here}.

\section{Notation}

%\wss{You should describe your notation.  You can use what is below as a starting point.}

The structure of the MIS for modules comes from [HoffmanAndStrooper1995],
with the addition that template modules have been adapted from
[GhezziEtAl2003].  The mathematical notation comes from Chapter 3 of
[HoffmanAndStrooper1995].  For instance, the symbol := is used for a
multiple assignment statement and conditional rules follow the form $(c_1
\Rightarrow r_1 | c_2 \Rightarrow r_2 | ... | c_n \Rightarrow r_n )$.

The following table summarizes the primitive data types used by BC. 
\begin{center}
\renewcommand{\arraystretch}{1.2}
\noindent 
\begin{tabular}{l l p{7.5cm}} 
\toprule 
\textbf{Data Type} & \textbf{Notation} & \textbf{Description}\\ 
\midrule
character & char & a single symbol or digit\\
integer & $\mathbb{Z}$ & a number without a fractional component in (-$\infty$, $\infty$) \\
natural number & $\mathbb{N}$ & a number without a fractional component in [1, $\infty$) \\
real & $\mathbb{R}$ & any number in (-$\infty$, $\infty$)\\
empty & $\varnothing$ & when the input is empty or the variable does not exist \\
GeoDataFrame & GeoDataFrame &  pandas dataFrame object with geometry column\\
\bottomrule
\end{tabular} 
\end{center}

\noindent
The specification of BC uses some derived data types: sequences, strings, and
tuples. Sequences are lists filled with elements of the same data type. Strings
are sequences of characters. Tuples contain a list of values, potentially of
different types. In addition, BC uses functions, which
are defined by the data types of their inputs and outputs. Local functions are
described by giving their type signature followed by their specification.

\section{Module Decomposition}

The following table is taken directly from the Module Guide document for this project.
\begin{table}[h!]
\centering
\begin{tabular}{p{0.3\textwidth} p{0.6\textwidth}}
\toprule
\textbf{Level 1} & \textbf{Level 2}\\
\midrule

{Hardware-Hiding Module} & ~ \\
\midrule

\multirow{14}{0.3\textwidth}{Behaviour-Hiding Module} & Input Module\\
& Control Module\\
& Data Searching Module\\
& Output Visualization Module\\
& Data Model Reading Module \\
& Constant Module \\
& Deicing Salts Calculation Module \\
& Melted Water Thickness Calculation Module \\
& Single Water SAS Calculation Module \\
& Single Chloride Ions SAS Calculation Module \\
& All Chloride Ions SAS Calculation Module \\
& Chloride on Pier Calculation Module \\
& Chloride on Deck Calculation Module \\
& Chloride Exposure Database Generation Module\\
\midrule

\multirow{2}{0.3\textwidth}{Software Decision Module} &  File I/O Module \\
& Plotting Module \\
\bottomrule

\end{tabular}
\caption{Module Hierarchy}
\label{TblMH}
\end{table}
\newpage
~\newpage

\section{MIS of Control Module} \label{controlModule} 
This module provides the main program and the GUI of the software.
%\wss{You can reference SRS labels, such as R\ref{R_Inputs}.}

%\wss{It is also possible to use \LaTeX for hypperlinks to external documents.}

\subsection{Module}

app

\subsection{Uses}
\begin{itemize}
\item Input Module (Section \ref{inputModule})
\item Data Searching Module (Section \ref{dataSearchingModule})
\item Output Visualization Module (Section \ref{outputVisualizationModule})
\end{itemize}


\subsection{Syntax}

\subsubsection{Exported Constants}

None

\subsubsection{Exported Access Programs}

\begin{center}
\begin{tabular}{p{2cm} p{5cm} p{4cm} p{2cm}}
\hline
\textbf{Name} & \textbf{In} & \textbf{Out} & \textbf{Exceptions} \\
\hline
main & -  & - & - \\
\hline
\end{tabular}
\end{center}

\subsection{Semantics}

\subsubsection{State Variables}
None
%\wss{Not all modules will have state variables.  State variables give the module a memory.}

\subsubsection{Environment Variables}

None

%\wss{This section is not necessary for all modules.  Its purpose is to capture when the module has external interaction with the environment, such as for a device driver, screen interface, keyboard, file, etc.}

\subsubsection{Assumptions}

None
%\wss{Try to minimize assumptions and anticipate programmer errors via exceptions, but for practical purposes assumptions are sometimes appropriate.}

\subsubsection{Access Routine Semantics}

\noindent app():
\begin{itemize}
\item transition: Control and execute the other modules as follow:
\begin{itemize}
\item Get and verify the input from user. (Section \ref{inputModule}) 
\item Search the corresponding data in Data Searching Module if the input is valid. (Section \ref{dataSearchingModule})
\item Visualize the resulting data by using Output Visualization Module. (Section \ref{outputVisualizationModule})

\end{itemize}
\item output: out := None
\item exception: exc := None
\end{itemize}

%\wss{A module without environment variables or state variables is unlikely to have a state transition.  In this case a state transition can only occur if the module is changing the state of another module.}

%\wss{Modules rarely have both a transition and an output.  In most cases you will have one or the other.}

\subsubsection{Local Functions}
None
%\wss{As appropriate} \wss{These functions are for the purpose of specification. They are not necessarily something that is going to be implemented explicitly.  Even if they are implemented, they are not exported; they only have local scope.}
  
~\newpage

\section{MIS of Input Module} \label{inputModule} 
This module get the input from user and verify if it is within the physical and software constraints.
\subsection{Module}
CoordinateChecker
\subsection{Uses}
None

\subsection{Syntax}

\subsubsection{Exported Constants}
None

\subsubsection{Exported Access Programs}

\begin{center}
\begin{tabular}{p{4cm} p{3.5cm} p{1.5cm} p{4.5cm}}
\hline
\textbf{Name} & \textbf{In} & \textbf{Out} & \textbf{Exceptions} \\
\hline
convertLongitude & long: String & long: $\mathbb{R}$ & ValueError \\
convertLatitude & lat: String & lat: $\mathbb{R}$ & ValueError \\
checkCoordinate & long: $\mathbb{R}$, lat: $\mathbb{R}$ & Boolean & InputOutofOntarioError \\

\hline
\end{tabular}
\end{center}

\subsection{Semantics}

\subsubsection{State Variables}
\begin{itemize}
\item lon: $\mathbb{R}$ \# longitude get from user.
\item lat: $\mathbb{R}$ \# latitude get from user.
\item isWithinOntario: Boolean \# if the point is within Ontario
\item selectedOption: String \# the component that user wants to investigate
\item rateOption: String \# the salt application rate that user wants to investigate
\end{itemize}
\subsubsection{Environment Variables}

\begin{itemize}
\item Keyboard: takes input from the keyboard by typing.
\item File: the geojson file that contains the shape of Ontario.
\end{itemize}

\subsubsection{Assumptions}
This module use the open source geojson file that contain the Ontario boundary, by drawing polygons with the vertex coordinate. The coordinates for those vertexes are assumed to be reliable.

\subsubsection{Access Routine Semantics}

\noindent \textbf{convertLongitude($long$)}:
\begin{itemize}
\item output: out := long: $\mathbb{R}$
\item exception: exc := ValueError: isNaN(parseFloat($long$))
\end{itemize}

\noindent \textbf{convertLatitude($lat$)}:
\begin{itemize}
\item output: out := lat: $\mathbb{R}$
\item exception: exc := ValueError: isNaN(parseFloat($lat$))
\end{itemize}

\noindent \textbf{checkCoordinate($long, lat$)}:
\begin{itemize}
\item output: out := isWithinOntario: Boolean
\item exception: exc := InputOutofOntarioError: $\lnot ((long, lat) \in Ontario)$
\end{itemize}

\subsubsection{Local Functions}
\noindent isNaN($e$):
\begin{itemize}
\item output: out := Boolean \# check if the input $e$ is NaN 
\end{itemize}
  
\noindent parseFloat($s$):
\begin{itemize}
\item output: out := s: $\mathbb{R}$ \# convert the input $s$ from type string to float, if it is not number and can not be converted, it would return NaN 
\end{itemize}

\noindent InputOutofOntarioError:
\begin{itemize}
\item output: out := Exception \# raise this exception if the input is out of Ontario 
\end{itemize}
  
~\newpage

\section{MIS of Data Searching Module} \label{dataSearchingModule} 
This module finds the data needed in the database.
\subsection{Module}

DataSearching

\subsection{Uses}
 Input Module (Section \ref{inputModule}), Chloride Exposure Database Generation Module (Section \ref{calculationResultModule})



\subsection{Syntax}

\subsubsection{Exported Constants}
None
\subsubsection{Exported Access Programs}

\begin{center}
\begin{tabular}{p{2cm} p{4cm} p{4cm} p{4cm}}
\hline
\textbf{Name} & \textbf{In} & \textbf{Out} & \textbf{Exceptions} \\
\hline
loadData & dataOption: String, rateOption: String & filePath: String & - \\
findClosest & long: $\mathbb{R}$, lat: $\mathbb{R}$ & index: $\mathbb{N}$   & - \\
searchData & long: $\mathbb{R}$, lat: $\mathbb{R}$  & data: sequence of  $\mathbb{R}$ & - \\
\hline
\end{tabular}
\end{center}

\subsection{Semantics}

\subsubsection{State Variables}
\begin{itemize}
\item data: sequence of $\mathbb{R}$ \# the sequence of predicted chloride exposure data that the user want

\end{itemize}

\subsubsection{Environment Variables}
\begin{itemize}
\item File: the database file that contain the yearly chloride exposure data within Ontario.
\end{itemize}

\subsubsection{Assumptions}
All locations within Ontario, if it is not on water, must contain valid data. 

\subsubsection{Access Routine Semantics}
\noindent \textbf{loadData($dataOption, rateOption$)}:
\begin{itemize}
\item output: out := filePath \# filePath for the data that user want
\item exception: exc := FileNotFoundError = $\nexists ~ filename$
\end{itemize}
\noindent \textbf{findClosest($long, lat$)}: If the input coordinate is not the exact one in database, find the grid that it belongs to and return the index of center coordinate.
\begin{itemize}
\item output: out := index: $\mathbb{N}$
\end{itemize}
\noindent \textbf{searchData($long, lat$)}:
\begin{itemize}
\item output: out := data = search(findClosest($long, lat$)) \# sequence of $\mathbb{R}$, the chloride exposure result
\item exception: exc := None
\end{itemize}

\subsubsection{Local Functions}
None

~\newpage

\section{MIS of Output Visualization Module} \label{outputVisualizationModule}
This module provides the visualization of the resulting chloride exposure trend.

\subsection{Module}

dataVisualization

\subsection{Uses}

Data Searching Module (Section \ref{dataSearchingModule})


\subsection{Syntax}

\subsubsection{Exported Constants}
None
\subsubsection{Exported Access Programs}

\begin{center}
\begin{tabular}{p{4cm} p{4cm} p{2cm} p{2cm}}
\hline
\textbf{Name} & \textbf{In} & \textbf{Out} & \textbf{Exceptions} \\
\hline
generateChartData & data: sequence of $\mathbb{R}$ & - & - \\
downloadData & - & - & - \\
downloadJPG & - & - & - \\
\hline
\end{tabular}
\end{center}

\subsection{Semantics}

\subsubsection{State Variables}
None

\subsubsection{Environment Variables}
\begin{itemize}
\item Screen: The graphs are displayed on the screen.
\item File: The files that could be downloaded to the local machine
\end{itemize}
\subsubsection{Assumptions}
None
\subsubsection{Access Routine Semantics}

\noindent \textbf{generateChartData($data$)}:
\begin{itemize}
\item transition: display the graphs using the result data from Data Searching Module.
\item output: out := None
\item exception: exc := None
\end{itemize}

\noindent \textbf{downloadData()}:
\begin{itemize}
\item transition: download the data to the local machine
\item output: out := None
\item exception: exc := None
\end{itemize}

\noindent \textbf{downloadJPG()}:
\begin{itemize}
\item transition: download the visualization to the local machine
\item output: out := None
\item exception: exc := None
\end{itemize}

\subsubsection{Local Functions}
None
\newpage


\section{MIS of Data Model Reading Module} \label{dataModelReadingModule}
This module loads the climate and traffic data from the external file and store it in the data format that could be used for calculation.

\subsection{Module}
calculation\_loadT

\subsection{Uses}

None


\subsection{Syntax}

\subsubsection{Exported Constants}
None
\subsubsection{Exported Access Programs}

\begin{center}
\begin{tabular}{p{4cm} p{3cm} p{4cm} p{4cm}}
\hline
\textbf{Name} & \textbf{In} & \textbf{Out} & \textbf{Exceptions} \\
\hline
new calculation\_load & filename: String & calculation\_loadT & FileNotFoundError \\
\hline
\end{tabular}
\end{center}

\subsection{Semantics}
A data structure designed to store the data from climate and traffic model.
\subsubsection{State Variables}
\begin{itemize}

\item long: sequence of $\mathbb{R}$ \# longitude get from climate and traffic model
\item lat: sequence of $\mathbb{R}$ \# latitude get from climate and traffic model
\item AADT: sequence of $\mathbb{R}$ \# annual average daily traffic per lane
\item AADTT: sequence of $\mathbb{R}$ \# annual average daily truck traffic per lane
\item t1: sequence of $\mathbb{R}$  \# number of days with snowfall
\item h\_total: sequence of $\mathbb{R}$ \# the total snowfall during a winter season
\item t2: sequence of $\mathbb{R}$  \# number of days with snow melting
\end{itemize}

\subsubsection{Environment Variables}
File: the file with all climate model data and traffic model data
\subsubsection{Assumptions}
None
\subsubsection{Access Routine Semantics}

\noindent \textbf{calculation\_load}:
\begin{itemize}
\item transition: Read and store the data from the climate model and traffic model file
\item output: out := self
\item exception: exc := FileNotFoundError = $\nexists ~ filename$
\end{itemize}

\subsubsection{Local Functions}
None
\newpage


\section{MIS of Constant Module} \label{constantModule}
This module stores the constants used for calculation. 

\subsection{Module}
constantT

\subsection{Uses}

None


\subsection{Syntax}

\subsubsection{Exported Constants}

\begin{center}
\begin{tabular}{p{4cm} p{1.5cm} p{11cm}}
\hline
\textbf{Name} & \textbf{Value} & \textbf{Note} \\
\hline
salt\_application\_rate & 0.07 & salt application rate\\
W\_lane & 3.75 & lane width\\
V\_speed & 100 & heavy vehicle speed\\
b & 0.56 & tire width\\
K & 0.75 & ratio of the tire width that is not a groove to the tire width\\
h\_film & 0.0001 & depth of the water film picked up in each rotation\\
water\_density & 997 & density of water\\
V & 62.1371 & truck speed\\
chloride\_ratio & 0.61 & molar mass ratio of chloride ions over deicing salts\\
d & 3.5 & distance between road edge and nearby bridge structure\\
ldv\_ratio & 6 & ratio of chloride ions sprayed and splashed by trucks over light-duty vehicles\\
\hline
\end{tabular}
\end{center}

\subsubsection{Exported Access Programs}
None

\subsection{Semantics}
\subsubsection{State Variables}
None
\subsubsection{Environment Variables}
None
\subsubsection{Assumptions}
None
\subsubsection{Access Routine Semantics}
None

\subsubsection{Local Functions}
None
\newpage

\section{MIS of Deicing Salts Calculation Module} \label{deicingSaltsCalculationModule}
This module provides the calculation for the quantity of deicing salts applied per day with snowfall

\subsection{Module}
deicing\_salts\_cal

\subsection{Uses}

Constant Module (Section \ref{constantModule}), Data Model Reading Module (Section \ref{dataModelReadingModule})


\subsection{Syntax}

\subsubsection{Exported Constants}
None
\subsubsection{Exported Access Programs}

\begin{center}
\begin{tabular}{p{3.5cm} p{4.5cm} p{4.5cm} p{2cm}}
\hline
\textbf{Name} & \textbf{In} & \textbf{Out} & \textbf{Exceptions} \\
\hline
deicing\_salts\_cal & h\_total: sequence of $\mathbb{R}$, t1: sequence of $\mathbb{R}$ & M\_app: sequence of $\mathbb{R}$ & - \\
\hline
\end{tabular}
\end{center}

\subsection{Semantics}

\subsubsection{State Variables}
None
\subsubsection{Environment Variables}
None
\subsubsection{Assumptions}
None
\subsubsection{Access Routine Semantics}
\noindent \textbf{deicing\_salts\_cal($h\_total, t1$)}:
\begin{itemize}
\item transition: None
\item output: out :=  $\frac{salt\_application\_rate*h\_total}{ (W\_lane*t1)}$, where salt\_application\_rate and W\_lane are constant value get from Constant Module, h\_total and t1 are read from Data Model Reading Module
\item exception: exc := None
\end{itemize}
\subsubsection{Local Functions}
None
\newpage


\section{MIS of Melted Water Thickness Module} \label{meltedWaterThicknessModule}
This module provides the calculation for the daily water film thickness on the road

\subsection{Module}
melted\_water\_thickness

\subsection{Uses}
Data Model Reading Module (Section \ref{dataModelReadingModule})


\subsection{Syntax}

\subsubsection{Exported Constants}
None
\subsubsection{Exported Access Programs}

\begin{center}
\begin{tabular}{p{4cm} p{4.5cm} p{4.5cm} p{2cm}}
\hline
\textbf{Name} & \textbf{In} & \textbf{Out} & \textbf{Exceptions} \\
\hline
melted\_water\_thickness & h\_total: sequence of $\mathbb{R}$, t2: sequence of $\mathbb{R}$ & h\_app: sequence of $\mathbb{R}$ & - \\

\hline
\end{tabular}
\end{center}

\subsection{Semantics}

\subsubsection{State Variables}
None
\subsubsection{Environment Variables}
None
\subsubsection{Assumptions}
None
\subsubsection{Access Routine Semantics}

\noindent \textbf{melted\_water\_thickness($h\_total, t2$)}:
\begin{itemize}
\item transition: None
\item output: out := $\frac{h\_total}{t2}$, where h\_total and t1 are read from Data Model Reading Module
\item exception: exc := None
\end{itemize}
\subsubsection{Local Functions}
None
\newpage

\section{MIS of Single Water SAS Calculation Module} \label{singleWaterSASCalculationModule}
This module determine water sprayed and splashed by one truck using a  (CFD)-based analytical model, taking into account of the four primary mechanisms of vehicle spray and splash: capillary adhesion, tread pickup, bow wave, and side wave.

\subsection{Module}
single\_water\_SAS\_cal

\subsection{Uses}

Constant Module (Section \ref{constantModule}), Melted Water Thickness Module (Section \ref{meltedWaterThicknessModule})


\subsection{Syntax}

\subsubsection{Exported Constants}
None
\subsubsection{Exported Access Programs}

\begin{center}
\begin{tabular}{p{4cm} p{4.5cm} p{4.5cm} p{2cm}}
\hline
\textbf{Name} & \textbf{In} & \textbf{Out} & \textbf{Exceptions} \\
\hline

single\_water\_SAS\_cal & h\_app: sequence of $\mathbb{R}$ & SD\_total: sequence of $\mathbb{R}$ & - \\
\hline
\end{tabular}
\end{center}

\subsection{Semantics}

\subsubsection{State Variables}
None
\subsubsection{Environment Variables}
None
\subsubsection{Assumptions}
None
\subsubsection{Access Routine Semantics}

\noindent \textbf{single\_water\_SAS\_cal($h\_app$)}:
\begin{itemize}
\item transition: None
\item output: out := $\mathit{SD_{CA}}$ + $\mathit{SD_{TP}}$ + $\mathit{SD_{BW}}$ + $\mathit{SD_{SW}}$ \# the mass of water per unit air volume kicked up by each passing truck is the sum of the four mechanisms, calculated by the local functions below.
\item exception: exc := None
\end{itemize}
\subsubsection{Local Functions}
$V_{speed}, b, K, h_{film}, \rho_{water}, V'$ are constants read from Constant Module.\\
\noindent \textbf{mass\_flow\_rate($h\_app$)}:
\begin{itemize}
\item transition: None
\item output: out := \begin{equation}
     \begin{cases}
     \mathit{MR_{CA}} = V_{speed} \times b \times K \times h_{film} \times \rho_{water} & \text{for} ~ CA \\
      \mathit{MR_{TP}} = V_{speed} \times b \times (1-K) \times h_{app} \times \rho_{water} & \text{for} ~ TP\\
      \mathit{MR_{BW}} = \mathit{MR_{SW}} = 0.5 \times V_{speed} \times b \times \\ (h_{app} - K \times h_{film} - (1-K) \times h_{app}) \times \rho_{water} & \text{for} ~ BW ~ and~ SW \\
      \end{cases}\nonumber
  \end{equation}
  
\item exception: exc := None
\end{itemize}
\noindent \textbf{spray\_density($\mathit{MR_{CA}}, \mathit{MR_{TP}}, \mathit{MR_{BW}}, \mathit{MR_{SW}}$)}:
\begin{itemize}
\item transition: None
\item output: out := 
\begin{equation}
     \begin{cases}
     \mathit{SD_{CA}} = (-2.69 \times 10^{-5} \times V' + 2.43 \times 10^{-3}) \times \mathit{MR_{CA}} & \text{for} ~ CA \\
      \mathit{SD_{TP}} = (1.16 \times 10^{-5} \times V' - 5.25 \times 10^{-5}) \times \mathit{MR_{TP}} & \text{for} ~ TP\\      
      \mathit{SD_{BW}} = (2.67 \times 10^{-5} \times V' - 4.71 \times 10^{-4}) \times \mathit{MR_{BW}} & \text{for} ~ BW\\
       \mathit{SD_{SW}} = (1.65 \times 10^{-5} \times V' - 3.99 \times 10^{-4}) \times \mathit{MR_{SW}} & \text{for} ~ SW\\      
      \end{cases}\nonumber
  \end{equation}
\item exception: exc := None
\end{itemize}
\subsubsection{Local Functions}
None
\newpage


\section{MIS of Single Chloride Ions SAS Calculation Module} \label{singleChlorideIonsSASCalculationModule}
This module determines the chloride ions sprayed and splashed by one truck.

\subsection{Module}
single\_Cl\_SAS\_cal

\subsection{Uses}

Deicing Salts Calculation Module (Section \ref{deicingSaltsCalculationModule}), Single Water SAS Calculation Module (Section \ref{singleWaterSASCalculationModule})


\subsection{Syntax}

\subsubsection{Exported Constants}
None
\subsubsection{Exported Access Programs}

\begin{center}
\begin{tabular}{p{3cm} p{4.5cm} p{5cm} p{2cm}}
\hline
\textbf{Name} & \textbf{In} & \textbf{Out} & \textbf{Exceptions} \\
\hline
single\_Cl\_SAS\_cal & M\_app: sequence of $\mathbb{R}$, h\_app: sequence of $\mathbb{R}$, SD\_total: sequence of $\mathbb{R}$ & SD\_totalCl: sequence of $\mathbb{R}$ & - \\

\hline
\end{tabular}
\end{center}

\subsection{Semantics}

\subsubsection{State Variables}
None

\subsubsection{Environment Variables}
None
\subsubsection{Assumptions}
None
\subsubsection{Access Routine Semantics}

\noindent \textbf{single\_Cl\_SAS\_cal($M\_app, h\_app, SD\_total$)}:
\begin{itemize}
\item transition: None
\item output: out := $SD\_total * salt\_water\_ratio($M\_app, h\_app$) * chloride\_ratio$, where chloride\_ratio is a constant read from Constant Module.
\item exception: exc := None
\end{itemize}

\subsubsection{Local Functions}
\noindent \textbf{salt\_water\_ratio($M\_app, h\_app$)}:
\begin{itemize}
\item transition: None
\item output: out := $\frac{M\_app}{h\_app*water\_density}$ where water\_density is a constant read from Constant Module.
\item exception: exc := None
\end{itemize}

\newpage

\section{MIS of All Chloride Ions SAS Calculation Module} \label{allChlorideIonsSASCalculationModule}
This module determines chloride ions sprayed and splashed by all vehicles in one winter season

\subsection{Module}
all\_Cl\_SAS\_cal

\subsection{Uses}
Constant Module (Section \ref{constantModule}), Data Model Reading Module (Section \ref{dataModelReadingModule}), Single Water SAS Calculation Module (Section \ref{singleWaterSASCalculationModule})


\subsection{Syntax}

\subsubsection{Exported Constants}
None
\subsubsection{Exported Access Programs}

\begin{center}
\begin{tabular}{p{2.5cm} p{4.5cm} p{5cm} p{2cm}}
\hline
\textbf{Name} & \textbf{In} & \textbf{Out} & \textbf{Exceptions} \\
\hline
all\_Cl\_SAS\_cal & SD\_totalCl: sequence of $\mathbb{R}$, t2: sequence of $\mathbb{R}$, AADT: sequence of $\mathbb{R}$, AADTT: sequence of $\mathbb{R}$ & C\_s\_air: sequence of $\mathbb{R}$ & - \\
updateAADT & AADT: sequence of $\mathbb{R}$ & AADT: sequence of $\mathbb{R}$ & - \\
updateAADTT & AADTT: sequence of $\mathbb{R}$ & AADTT: sequence of $\mathbb{R}$ & - \\
\hline
\end{tabular}
\end{center}

\subsection{Semantics}

\subsubsection{State Variables}
None

\subsubsection{Environment Variables}
None
\subsubsection{Assumptions}
The AADT and AADTT are assumed to have 2\% increase rate every year
\subsubsection{Access Routine Semantics}

\noindent \textbf{all\_Cl\_SAS\_cal($M\_app, h\_app, SD\_total, t2, AADT, AADTT$)}:
\begin{itemize}
\item transition: None
\item output: out := $(\frac{SD\_totalCl}{ldv\_ratio}*(updateAADT($AADT$)-updateAADTT($AADTT$))+ SD\_totalCl* AADTT) * t2$, where ldv\_ratio is a constant read from Constant Module.
\item exception: exc := None
\end{itemize}
\noindent \textbf{updateAADT($AADT$)}:
\begin{itemize}
\item transition: None
\item output: out := AADT \# calculate the AADT for future year, assuming a 2\% annual increase rate 
\item exception: exc := None
\end{itemize}
\noindent \textbf{updateAADTT($AADTT$)}:
\begin{itemize}
\item transition: None
\item output: out := AADTT \# calculate the AADTT for future year, assuming a 2\% annual increase rate 
\item exception: exc := None
\end{itemize}


\subsubsection{Local Functions}
None

\newpage


\section{MIS of Chloride on Pier Calculation Module} \label{chlorideonPierCalculationModule}
This module determine the deposition of chloride ions on the pier of the bridge substructure
\subsection{Module}
chloride\_on\_pier

\subsection{Uses}

Constant Module (Section \ref{constantModule}), All Chloride Ions SAS Calculation Module (Section \ref{allChlorideIonsSASCalculationModule})

\subsection{Syntax}

\subsubsection{Exported Constants}
None
\subsubsection{Exported Access Programs}

\begin{center}
\begin{tabular}{p{4cm} p{4.5cm} p{4cm} p{2cm}}
\hline
\textbf{Name} & \textbf{In} & \textbf{Out} & \textbf{Exceptions} \\
\hline
chloride\_on\_pier & C\_s\_air: sequence of $\mathbb{R}$ & results: sequence of $\mathbb{R}$ & - \\

\hline
\end{tabular}
\end{center}

\subsection{Semantics}

\subsubsection{State Variables}
None

\subsubsection{Environment Variables}
None
\subsubsection{Assumptions}
None
\subsubsection{Access Routine Semantics}

\noindent \textbf{chloride\_\_on\_pier($C\_s\_air$)}:
\begin{itemize}
\item transition: None
\item output: out := $C\_s\_air * 0.015 * e^{-0.05*d}  + C\_s\_air*0.985 * e^{-0.5*d}$, where d is a constant read from Constant Module, 0.015 and 0.985 being the coefficient of the formula.
\item exception: exc := None
\end{itemize}

\subsubsection{Local Functions}
None
\newpage


\section{MIS of Chloride on Deck Calculation Module} \label{chlorideonDeckCalculationModule}
This module determine the deposition of chloride ions on the deck of the bridge substructure
\subsection{Module}
chloride\_on\_deck

\subsection{Uses}
Data Model Reading Module (Section \ref{dataModelReadingModule})

\subsection{Syntax}

\subsubsection{Exported Constants}
None
\subsubsection{Exported Access Programs}

\begin{center}
\begin{tabular}{p{4cm} p{4.5cm} p{4cm} p{2cm}}
\hline
\textbf{Name} & \textbf{In} & \textbf{Out} & \textbf{Exceptions} \\
\hline
chloride\_on\_deck & h\_total: sequence of $\mathbb{R}$, AADT: sequence of $\mathbb{R}$ & results: sequence of $\mathbb{R}$ & - \\

\hline
\end{tabular}
\end{center}

\subsection{Semantics}

\subsubsection{State Variables}
None

\subsubsection{Environment Variables}
None
\subsubsection{Assumptions}
None
\subsubsection{Access Routine Semantics}

\noindent \textbf{chloride\_on\_deck(h\_total, AADT)}:
\begin{itemize}
\item transition: None
\item output: out := $0.11*h\_total - 0.000189*AADT + 3.349$. This is a linear regression model.
\item exception: exc := None
\end{itemize}

\subsubsection{Local Functions}
None
\newpage

\section{MIS of Chloride Exposure Database Generation Module} \label{calculationResultModule}
This module performs the calculation process to generate the database, it is related to the R2, R3 in SRS.


\subsection{Module}

calculate

\subsection{Uses}

Data Model Reading Module (Section \ref{dataModelReadingModule}), 
Constant Module (Section \ref{constantModule}), Deicing Salts Calculation Module (Section \ref{deicingSaltsCalculationModule}), Melted Water Thickness Module (Section \ref{meltedWaterThicknessModule}), Single Water SAS Calculation Module (Section \ref{singleWaterSASCalculationModule}), Single Chloride Ions SAS Calculation Module (Section \ref{singleChlorideIonsSASCalculationModule}), All Chloride Ions SAS Calculation Module (Section \ref{allChlorideIonsSASCalculationModule}), Chloride on Pier Calculation Module (Section \ref{chlorideonPierCalculationModule}),  Chloride on Deck Calculation Module (Section \ref{chlorideonDeckCalculationModule})


\subsection{Syntax}

\subsubsection{Exported Constants}
None
\subsubsection{Exported Access Programs}

\begin{center}
\begin{tabular}{p{2cm} p{4.5cm} p{4cm} p{4cm}}
\hline
\textbf{Name} & \textbf{In} & \textbf{Out} & \textbf{Exceptions} \\
\hline
calculate & AADT: sequence of $\mathbb{R}$, AADTT: sequence of $\mathbb{R}$, t1: sequence of $\mathbb{R}$, h\textsubscript{total}: sequence of $\mathbb{R}$, t2: sequence of $\mathbb{R}$ & result: sequence of $\mathbb{R}$ & DataMissingError, DataInvalidError \\

savefile & long: sequence of $\mathbb{R}$, lat: sequence of $\mathbb{R}$, results: sequence of $\mathbb{R}$ & file: String & - \\
\hline
\end{tabular}
\end{center}

\subsection{Semantics}


\subsubsection{State Variables}
None
\subsubsection{Environment Variables}
File: the result of calculation will be stored in an output csv file.
\subsubsection{Assumptions}
The map of Ontario is divided into multiple 25km * 25km grid (as mentioned in SRS) and the coordinates are the center of those grids. The locations inside each grid are consider to have same chloride exposure rate.

\subsubsection{Access Routine Semantics}

\noindent \textbf{calculate($AADT, AADTT, t1, h_{total}, t2$)}:
\begin{itemize}
\item transition: use all the formulas from calculate\_step1 to calculate\_step6, conclude the final result
\item output: out := result \# Sequence of $\mathbb{R}$
\item exception: exc:= \\ \\ 
 \begin{tabular}{p{10cm} p{3.5cm} }
 \hline
 \textbf{Expression} & \textbf{Exception}  \\
 \hline
     $\exists e \in [AADT, AADTT, h_{total}, t1, t2], e =\varnothing$ & DataMissingError  \\ \\

  \hline
     $(\exists i \in [0..|AADT|-1], AADTT[i] > AADT[i]) \lor (\neg (t1, t2 \in (0,365)))$   & DataInvalidError \\ \\

  \hline
 \end{tabular}

\end{itemize}


\noindent \textbf{savefile($long, lat, results$)}:
\begin{itemize}
\item transition: Save the longitude, latitude and the corresponding results for each grid to a csv file, which is the prediction of chloride exposure rate. The file has a row label as coordinate and a column label as year. 
\item output: out := file
\item exception: exc := None
\end{itemize}



\subsubsection{Local Functions}
None
\bibliographystyle {plainnat}
\bibliography {../../../refs/References}

\newpage



\end{document}