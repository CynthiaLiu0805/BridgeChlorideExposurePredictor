\documentclass[12pt, titlepage]{article}

\usepackage{amsmath, mathtools}

\usepackage[round]{natbib}
\usepackage{amsfonts}
\usepackage{amssymb}
\usepackage{graphicx}
\usepackage{colortbl}
\usepackage{xr}
\usepackage{hyperref}
\usepackage{longtable}
\usepackage{xfrac}
\usepackage{tabularx}
\usepackage{float}
\usepackage{siunitx}
\usepackage{booktabs}
\usepackage{multirow}
\usepackage[section]{placeins}
\usepackage{caption}
\usepackage{fullpage}

\hypersetup{
bookmarks=true,     % show bookmarks bar?
colorlinks=true,       % false: boxed links; true: colored links
linkcolor=red,          % color of internal links (change box color with linkbordercolor)
citecolor=blue,      % color of links to bibliography
filecolor=magenta,  % color of file links
urlcolor=cyan          % color of external links
}

\usepackage{array}

\externaldocument{../../SRS/SRS}

%% Comments

\usepackage{color}

\newif\ifcomments\commentstrue %displays comments
%\newif\ifcomments\commentsfalse %so that comments do not display

\ifcomments
\newcommand{\authornote}[3]{\textcolor{#1}{[#3 ---#2]}}
\newcommand{\todo}[1]{\textcolor{red}{[TODO: #1]}}
\else
\newcommand{\authornote}[3]{}
\newcommand{\todo}[1]{}
\fi

\newcommand{\wss}[1]{\authornote{blue}{SS}{#1}} 
\newcommand{\plt}[1]{\authornote{magenta}{TPLT}{#1}} %For explanation of the template
\newcommand{\an}[1]{\authornote{cyan}{Author}{#1}}

%% Common Parts

\newcommand{\progname}{ProgName} % PUT YOUR PROGRAM NAME HERE
\newcommand{\authname}{Team \#, Team Name
\\ Student 1 name
\\ Student 2 name
\\ Student 3 name
\\ Student 4 name} % AUTHOR NAMES                  

\usepackage{hyperref}
    \hypersetup{colorlinks=true, linkcolor=blue, citecolor=blue, filecolor=blue,
                urlcolor=blue, unicode=false}
    \urlstyle{same}
                                


\begin{document}

\title{Module Interface Specification for Bridge Corrosion}

\author{Cynthia Liu}

\date{\today}

\maketitle

\pagenumbering{roman}

\section{Revision History}

\begin{tabularx}{\textwidth}{p{3cm}p{2cm}X}
\toprule {\bf Date} & {\bf Version} & {\bf Notes}\\
\midrule
Mar. 8, 2024 & 1.0 & Initial Release\\
\bottomrule
\end{tabularx}

~\newpage

\section{Symbols, Abbreviations and Acronyms}

See SRS Documentation at \href{https://github.com/CynthiaLiu0805/BridgeCorrosion/blob/main/docs/SRS/SRS.pdf}{SRS}.

\wss{Also add any additional symbols, abbreviations or acronyms}

\newpage

\tableofcontents

\newpage

\pagenumbering{arabic}

\section{Introduction}

The following document details the Module Interface Specifications for Bridge Corrosion which investigate how climate, traffic might impact corrosion-induced
damage for reinforced concrete bridges by influencing the chloride exposure.
\wss{Fill in your project name and description}

Complementary documents include the System Requirement Specifications
and Module Guide.  The full documentation and implementation can be
found at \href{https://github.com/CynthiaLiu0805/BridgeCorrosion}{here}.

\section{Notation}

\wss{You should describe your notation.  You can use what is below as
  a starting point.}

The structure of the MIS for modules comes from [HoffmanAndStrooper1995],
with the addition that template modules have been adapted from
[GhezziEtAl2003].  The mathematical notation comes from Chapter 3 of
[HoffmanAndStrooper1995].  For instance, the symbol := is used for a
multiple assignment statement and conditional rules follow the form $(c_1
\Rightarrow r_1 | c_2 \Rightarrow r_2 | ... | c_n \Rightarrow r_n )$.

The following table summarizes the primitive data types used by BC. 
%TODO
\begin{center}
\renewcommand{\arraystretch}{1.2}
\noindent 
\begin{tabular}{l l p{7.5cm}} 
\toprule 
\textbf{Data Type} & \textbf{Notation} & \textbf{Description}\\ 
\midrule
character & char & a single symbol or digit\\
integer & $\mathbb{Z}$ & a number without a fractional component in (-$\infty$, $\infty$) \\
natural number & $\mathbb{N}$ & a number without a fractional component in [1, $\infty$) \\
real & $\mathbb{R}$ & any number in (-$\infty$, $\infty$)\\
\bottomrule
\end{tabular} 
\end{center}

\noindent
The specification of BC uses some derived data types: sequences, strings, and
tuples. Sequences are lists filled with elements of the same data type. Strings
are sequences of characters. Tuples contain a list of values, potentially of
different types. In addition, BC uses functions, which
are defined by the data types of their inputs and outputs. Local functions are
described by giving their type signature followed by their specification.

\section{Module Decomposition}

The following table is taken directly from the Module Guide document for this project.

\begin{table}[h!]
\centering
\begin{tabular}{p{0.3\textwidth} p{0.6\textwidth}}
\toprule
\textbf{Level 1} & \textbf{Level 2}\\
\midrule

{Hardware Hiding} & ~ \\
\midrule

\multirow{5}{0.3\textwidth}{Behaviour-Hiding Module} & Input Parameter Module\\
& Input Verification Module\\
& Control Module\\
& Data Searching Module\\
& Output Visualization Module\\

\midrule

\multirow{3}{0.3\textwidth}{Software Decision Module} & Chloride Exposure Calculation Model \\
& Sequence Data Structure Module \\
& Plotting Module \\
\bottomrule

\end{tabular}
\caption{Module Hierarchy}
\label{TblMH}
\end{table}

\newpage
~\newpage

\section{MIS of Control Model} \label{controlModule} \wss{Use labels for
  cross-referencing}

\wss{You can reference SRS labels, such as R\ref{R_Inputs}.}

\wss{It is also possible to use \LaTeX for hypperlinks to external documents.}

\subsection{Module}

main

\subsection{Uses}
\begin{itemize}
\item Hardware Hiding Module
\item Input Parameter Module
\item Input Verification Module
\item Data Searching Module
\item Output Visualization Module
\item Chloride Exposure Calculation Model 
\item Sequence Data Structure Module 
\item Plotting Module 
\end{itemize}


\subsection{Syntax}

\subsubsection{Exported Constants}

None

\subsubsection{Exported Access Programs}

\begin{center}
\begin{tabular}{p{2cm} p{5cm} p{4cm} p{2cm}}
\hline
\textbf{Name} & \textbf{In} & \textbf{Out} & \textbf{Exceptions} \\
\hline
main & - & - & - \\
\hline
\end{tabular}
\end{center}

\subsection{Semantics}

\subsubsection{State Variables}
long, lat: a pair of input of  [long: $\mathbb{R}$, lat: $\mathbb{R}$] \#input coordinate get from Input Parameter Module.
\wss{Not all modules will have state variables.  State variables give the module
  a memory.}

\subsubsection{Environment Variables}

None

\wss{This section is not necessary for all modules.  Its purpose is to capture
  when the module has external interaction with the environment, such as for a
  device driver, screen interface, keyboard, file, etc.}

\subsubsection{Assumptions}

None
\wss{Try to minimize assumptions and anticipate programmer errors via
  exceptions, but for practical purposes assumptions are sometimes appropriate.}

\subsubsection{Access Routine Semantics}

\noindent main():
\begin{itemize}
\item transition: Control and execute the other modules as follow:
\begin{itemize}
\item Get input from user by Input Parameter Module. (M2, Section \ref{inputParameterModule}) 
\item Pass the input to Input Verification Module. (M3, Section \ref{inputVerificationModule})
\item Search the corresponding data in Data Searching Module if the input is valid. (M5, Section \ref{dataSearchingModule})
\item Visualize the resulting data by using Output Visualization Module. (M6, Section \ref{outputVisualizationModule})

\end{itemize}
\item output: None
\item exception: Exception is handled in Section \ref{inputVerificationModule}
\end{itemize}

\wss{A module without environment variables or state variables is unlikely to
  have a state transition.  In this case a state transition can only occur if
  the module is changing the state of another module.}

\wss{Modules rarely have both a transition and an output.  In most cases you
  will have one or the other.}

\subsubsection{Local Functions}
None
\wss{As appropriate} \wss{These functions are for the purpose of specification.
  They are not necessarily something that is going to be implemented
  explicitly.  Even if they are implemented, they are not exported; they only
  have local scope.}
  
~\newpage

\section{MIS of Input Parameter Model} \label{inputParameterModule} \wss{Use labels for
  cross-referencing}

\wss{You can reference SRS labels, such as R\ref{R_Inputs}.}

\wss{It is also possible to use \LaTeX for hypperlinks to external documents.}

\subsection{Module}
param
\subsection{Uses}
None

\subsection{Syntax}

\subsubsection{Exported Constants}
None

\subsubsection{Exported Access Programs}

\begin{center}
\begin{tabular}{p{2cm} p{5cm} p{2cm} p{5cm}}
\hline
\textbf{Name} & \textbf{In} & \textbf{Out} & \textbf{Exceptions} \\
\hline
param & pair of [long: $\mathbb{R}$, lat: $\mathbb{R}$] & - & InputMissingError, InputTypeMismatchError \\
\hline
\end{tabular}
\end{center}

\subsection{Semantics}

\subsubsection{State Variables}
long, lat := a pair of  [long: $\mathbb{R}$, lat: $\mathbb{R}$] \#input coordinate
\wss{Not all modules will have state variables.  State variables give the module
  a memory.}

\subsubsection{Environment Variables}
Keyboard: this module takes input from the keyboard by typing.

\wss{This section is not necessary for all modules.  Its purpose is to capture
  when the module has external interaction with the environment, such as for a
  device driver, screen interface, keyboard, file, etc.}

\subsubsection{Assumptions}
None
\wss{Try to minimize assumptions and anticipate programmer errors via
  exceptions, but for practical purposes assumptions are sometimes appropriate.}

\subsubsection{Access Routine Semantics}
This module load the input data and save it to the data type needed by the Input Verification Module.\\
\noindent read\_input():
\begin{itemize}
\item transition: Get input from the user, pass the input to the Input Verification Module.
\item output: None
\item exception: None
\end{itemize}

\wss{A module without environment variables or state variables is unlikely to
  have a state transition.  In this case a state transition can only occur if
  the module is changing the state of another module.}

\wss{Modules rarely have both a transition and an output.  In most cases you
  will have one or the other.}

\subsubsection{Local Functions}

\wss{As appropriate} \wss{These functions are for the purpose of specification.
  They are not necessarily something that is going to be implemented
  explicitly.  Even if they are implemented, they are not exported; they only
  have local scope.}
  
  
~\newpage

\section{MIS of Input Verification Model} \label{inputVerificationModule} \wss{Use labels for cross-referencing}

\wss{You can reference SRS labels, such as R\ref{R_Inputs}.}

\wss{It is also possible to use \LaTeX for hypperlinks to external documents.}

\subsection{Module}

verify\_param

\subsection{Uses}
Input Parameter Module

\subsection{Syntax}

\subsubsection{Exported Constants}
None

\subsubsection{Exported Access Programs}

\begin{center}
\begin{tabular}{p{3cm} p{4cm} p{2cm} p{5cm}}
\hline
\textbf{Name} & \textbf{In} & \textbf{Out} & \textbf{Exceptions} \\
\hline
verify\_param & [long: $\mathbb{R}$, lat: $\mathbb{R}$] & Boolean & InputOutofOntarioError \\
\hline
\end{tabular}
\end{center}

\subsection{Semantics}

\subsubsection{State Variables}
\begin{itemize}
\item long, lat: a pair of [long: $\mathbb{R}$, lat: $\mathbb{R}$] \# input coordinate
\item isValid: Boolean \# if this input is valid
\item geojson: TODO
\end{itemize}
\wss{Not all modules will have state variables.  State variables give the module
  a memory.}

\subsubsection{Environment Variables}

\wss{This section is not necessary for all modules.  Its purpose is to capture
  when the module has external interaction with the environment, such as for a
  device driver, screen interface, keyboard, file, etc.}

\subsubsection{Assumptions}
This module use the open source geojson file TODO and the vertex coordinate there is assumed to be true.
\wss{Try to minimize assumptions and anticipate programmer errors via
  exceptions, but for practical purposes assumptions are sometimes appropriate.}

\subsubsection{Access Routine Semantics}

\noindent verify\_param(long, lat):
\begin{itemize}
\item output: Boolean
\item exception: InputOutofOntarioError
\end{itemize}

\wss{A module without environment variables or state variables is unlikely to
  have a state transition.  In this case a state transition can only occur if
  the module is changing the state of another module.}

\wss{Modules rarely have both a transition and an output.  In most cases you
  will have one or the other.}

\subsubsection{Local Functions}

\wss{As appropriate} \wss{These functions are for the purpose of specification.
  They are not necessarily something that is going to be implemented
  explicitly.  Even if they are implemented, they are not exported; they only
  have local scope.}
  
  
~\newpage

\section{MIS of Data Searching Model} \label{dataSearchingModule} \wss{Use labels for
  cross-referencing}

\wss{You can reference SRS labels, such as R\ref{R_Inputs}.}

\wss{It is also possible to use \LaTeX for hypperlinks to external documents.}

\subsection{Module}

search

\subsection{Uses}
 Input Parameter Module, Chloride Exposure Calculation Module

\subsection{Syntax}

\subsubsection{Exported Constants}

\subsubsection{Exported Access Programs}

\begin{center}
\begin{tabular}{p{2cm} p{4cm} p{4cm} p{4cm}}
\hline
\textbf{Name} & \textbf{In} & \textbf{Out} & \textbf{Exceptions} \\
\hline
read\_file & filename: String & data: sequence of $\mathbb{R}$ & FileNotFoundError \\
search & [long: $\mathbb{R}$, lat: $\mathbb{R}$]  & result: [cl: $\mathbb{R}$] & - \\
\hline
\end{tabular}
\end{center}

\subsection{Semantics}

\subsubsection{State Variables}
\begin{itemize}
\item long, lat: a pair of [long: $\mathbb{R}$, lat: $\mathbb{R}$] \# input coordinate
\item filename: String \# the database generated by Calculation Module 
\item data: sequence of $\mathbb{R}$ \# the sequence of data read from database
\item cl: $\mathbb{R}$ \# predicted chloride exposure data at one time, one location
\item result: [cl: $\mathbb{R}$] \# the list of predicted chloride exposure data

\end{itemize}
\wss{Not all modules will have state variables.  State variables give the module
  a memory.}

\subsubsection{Environment Variables}

\wss{This section is not necessary for all modules.  Its purpose is to capture
  when the module has external interaction with the environment, such as for a
  device driver, screen interface, keyboard, file, etc.}

\subsubsection{Assumptions}
All locations within Ontario must contain valid data. If a user inputs a location outside of Ontario, it will be handled by the Input Verification Module.
\wss{Try to minimize assumptions and anticipate programmer errors via
  exceptions, but for practical purposes assumptions are sometimes appropriate.}

\subsubsection{Access Routine Semantics}
\noindent read\_file(filename):
\begin{itemize}
\item transition: read the data in the database file
\item output: data := sequence of [cl: $\mathbb{R}$]
\item exception: FileNotFoundError := Could not find such file
\end{itemize}
\noindent search(long, lat, data):
\begin{itemize}
\item transition: if the input coordinate do not have a corresponding value in the data, (long, lat) := find\_grid\_belonged(long, lat, data). Use the new (long, lat) to do the searching.
\item output: result := [cl: $\mathbb{R}$]
\item exception: none
\end{itemize}

\wss{A module without environment variables or state variables is unlikely to
  have a state transition.  In this case a state transition can only occur if
  the module is changing the state of another module.}

\wss{Modules rarely have both a transition and an output.  In most cases you
  will have one or the other.}

\subsubsection{Local Functions}
\noindent \textbf{find\_grid\_belonged(long, lat, data)}: If the input coordinate is not the exact one in data, find the grid that it belongs to and return the center coordinate.
\begin{itemize}
\item output: (long, lat) := $\mathbb{R}$

\end{itemize}

\wss{As appropriate} \wss{These functions are for the purpose of specification.
  They are not necessarily something that is going to be implemented
  explicitly.  Even if they are implemented, they are not exported; they only
  have local scope.}

~\newpage

\section{MIS of Output Visualization Model} \label{outputVisualizationModule} \wss{Use labels for
  cross-referencing}

\wss{You can reference SRS labels, such as R\ref{R_Inputs}.}

\wss{It is also possible to use \LaTeX for hypperlinks to external documents.}

\subsection{Module}

draw

\subsection{Uses}

Data Searching Module

\subsection{Syntax}

\subsubsection{Exported Constants}

\subsubsection{Exported Access Programs}

\begin{center}
\begin{tabular}{p{2cm} p{4cm} p{7cm} p{2cm}}
\hline
\textbf{Name} & \textbf{In} & \textbf{Out} & \textbf{Exceptions} \\
\hline
draw & result: [cl: $\mathbb{R}$] & graphs & - \\
\hline
\end{tabular}
\end{center}

\subsection{Semantics}

\subsubsection{State Variables}
\begin{itemize}
\item long, lat: a pair of [long: $\mathbb{R}$, lat: $\mathbb{R}$] \# input coordinate
\item cl: $\mathbb{R}$ \# predicted chloride exposure data at one time, one location
\item result: [cl: $\mathbb{R}$] \# the list of predicted chloride exposure data
\end{itemize}

\wss{Not all modules will have state variables.  State variables give the module
  a memory.}

\subsubsection{Environment Variables}
Screen: The graphs are displayed on the screen.
\wss{This section is not necessary for all modules.  Its purpose is to capture
  when the module has external interaction with the environment, such as for a
  device driver, screen interface, keyboard, file, etc.}

\subsubsection{Assumptions}

\wss{Try to minimize assumptions and anticipate programmer errors via
  exceptions, but for practical purposes assumptions are sometimes appropriate.}

\subsubsection{Access Routine Semantics}

\noindent draw(long, lat, result):
\begin{itemize}
\item transition: display the graphs using the result from Data Searching Module.
\item output: None
\item exception: None
\end{itemize}

\wss{A module without environment variables or state variables is unlikely to
  have a state transition.  In this case a state transition can only occur if
  the module is changing the state of another module.}

\wss{Modules rarely have both a transition and an output.  In most cases you
  will have one or the other.}

\subsubsection{Local Functions}

\wss{As appropriate} \wss{These functions are for the purpose of specification.
  They are not necessarily something that is going to be implemented
  explicitly.  Even if they are implemented, they are not exported; they only
  have local scope.}
\newpage


\section{MIS of Chloride Exposure Calculation Model} \label{calculationModule} \wss{Use labels for
  cross-referencing}

\wss{You can reference SRS labels, such as R\ref{R_Inputs}.}

\wss{It is also possible to use \LaTeX for hypperlinks to external documents.}

\subsection{Module}

calculate

\subsection{Uses}

none

\subsection{Syntax}

\subsubsection{Exported Constants}

\subsubsection{Exported Access Programs}

\begin{center}
\begin{tabular}{p{2cm} p{4cm} p{4cm} p{4cm}}
\hline
\textbf{Name} & \textbf{In} & \textbf{Out} & \textbf{Exceptions} \\
\hline
calculate & AADT: sequence of $\mathbb{R}$, AADTT: sequence of $\mathbb{R}$, t1: sequence of $\mathbb{N}$, $h_{total}$: sequence of $\mathbb{R}$, t2: sequence of $\mathbb{N}$ & file: an output file & DataMissingError, DataInvalidError \\
\hline
\end{tabular}
\end{center}

\subsection{Semantics}


\subsubsection{State Variables}
none
\wss{Not all modules will have state variables.  State variables give the module
  a memory.}

\subsubsection{Environment Variables}
The result of calculation will be stored in an output csv file.
\wss{This section is not necessary for all modules.  Its purpose is to capture
  when the module has external interaction with the environment, such as for a
  device driver, screen interface, keyboard, file, etc.}

\subsubsection{Assumptions}

\begin{itemize}
\item The AADT and AADTT are assumed to have 2\% increase rate every year.
\item The map of Ontario is divided into multiple 25km * 25km grid (as mentioned in SRS) and the coordinates are the center of those grids. The locations inside each grid are consider to have same chloride exposure rate.
\end{itemize}
\wss{Try to minimize assumptions and anticipate programmer errors via
  exceptions, but for practical purposes assumptions are sometimes appropriate.}

\subsubsection{Access Routine Semantics}

\noindent calculate(AADT, AADTT, t1, $h_{total}$, t2):
\begin{itemize}
\item transition: result:= Cl\_mass(spray\_density\_Cl(salt\_per\_day($h_{total}$, t1), water\_thickness($h_{total}$, t2)), adjust(AADT, AADTT), t2). See Local Functions for step by step calculation.
\item output: The result is saved to a csv file storing calculation result, which is the prediction of chloride exposure rate. The file has row lable as coordinate and column lable as year. 
\item exception: exc:= \\ \\ 
 \begin{tabular}{p{10cm} p{3.5cm} }
 \hline
 \textbf{Expression} & \textbf{Exception}  \\
 \hline
     $\exists e \in [AADT, AADTT, h_{total}, t1, t2], e =\varnothing$ & DataMissingError  \\ \\

  \hline
     $(\exists i \in [0..|AADT|-1], AADTT[i] > AADT[i]) \lor (\neg (t1, t2 \in (0,365)))$   & DataInvalidError \\ \\

  \hline
 \end{tabular}

\end{itemize}

\wss{A module without environment variables or state variables is unlikely to
  have a state transition.  In this case a state transition can only occur if
  the module is changing the state of another module.}

\wss{Modules rarely have both a transition and an output.  In most cases you
  will have one or the other.}

\subsubsection{Local Functions}
\noindent \textbf{adjust(AADT, AADTT)}: A function that convert AADT and AADTT from sequence to 2-D array, based on the assumption of annually 2\% increase.
\begin{itemize}
\item out := AADT, AADTT
\end{itemize}

\noindent \textbf{salt\_per\_day(h\_total, t1)}: calcualte the quantity of deicing salts applied on a roadway per day during the winter season.
\begin{itemize}
\item out := M\_app
\end{itemize}

\noindent \textbf{water\_thickness(h\_total, t2)}: calculate the thickness of melted water per day with snow melting
\begin{itemize}
\item out := h\_app
\end{itemize}

\noindent \textbf{spray\_density\_Cl(M\_app, h\_app)}: calculate the mass of chloride ions by one truck
\begin{itemize}
\item out := SD\_totalCl
\end{itemize}

\noindent \textbf{Cl\_mass(SD\_totalCl, AADT, AADTT, t2)}: calculate the final chloride exposure
\begin{itemize}
\item out := Cl\_result
\end{itemize}


\wss{As appropriate} \wss{These functions are for the purpose of specification.
  They are not necessarily something that is going to be implemented
  explicitly.  Even if they are implemented, they are not exported; they only
  have local scope.}

\bibliographystyle {plainnat}
\bibliography {../../../refs/References}

\newpage



\end{document}