% THIS DOCUMENT IS TAILORED TO REQUIREMENTS FOR SCIENTIFIC COMPUTING.  IT SHOULDN'T
% BE USED FOR NON-SCIENTIFIC COMPUTING PROJECTS
\documentclass[12pt]{article}

\usepackage{amsmath, mathtools}
\usepackage{amsfonts}
\usepackage{amssymb}
\usepackage{graphicx}
\usepackage{colortbl}
\usepackage{xr}
\usepackage{hyperref}
\usepackage{longtable}
\usepackage{xfrac}
\usepackage{tabularx}
\usepackage{float}
\usepackage{siunitx}
\usepackage{booktabs}
\usepackage{caption}
\usepackage{pdflscape}
\usepackage{afterpage}

\usepackage{gensymb}

\usepackage[round]{natbib}

%\usepackage{refcheck}

\hypersetup{
    bookmarks=true,         % show bookmarks bar?
      colorlinks=true,       % false: boxed links; true: colored links
    linkcolor=red,          % color of internal links (change box color with linkbordercolor)
    citecolor=green,        % color of links to bibliography
    filecolor=magenta,      % color of file links
    urlcolor=cyan           % color of external links
}

\input{../Comments}
\input{../Common}

% For easy change of table widths
\newcommand{\colZwidth}{1.0\textwidth}
\newcommand{\colAwidth}{0.13\textwidth}
\newcommand{\colBwidth}{0.82\textwidth}
\newcommand{\colCwidth}{0.1\textwidth}
\newcommand{\colDwidth}{0.05\textwidth}
\newcommand{\colEwidth}{0.8\textwidth}
\newcommand{\colFwidth}{0.17\textwidth}
\newcommand{\colGwidth}{0.5\textwidth}
\newcommand{\colHwidth}{0.28\textwidth}

% Used so that cross-references have a meaningful prefix
\newcounter{defnum} %Definition Number
\newcommand{\dthedefnum}{GD\thedefnum}
\newcommand{\dref}[1]{GD\ref{#1}}
\newcounter{datadefnum} %Datadefinition Number
\newcommand{\ddthedatadefnum}{DD\thedatadefnum}
\newcommand{\ddref}[1]{DD\ref{#1}}
\newcounter{theorynum} %Theory Number
\newcommand{\tthetheorynum}{TM\thetheorynum}
\newcommand{\tref}[1]{TM\ref{#1}}
\newcounter{tablenum} %Table Number
\newcommand{\tbthetablenum}{TB\thetablenum}
\newcommand{\tbref}[1]{TB\ref{#1}}
\newcounter{assumpnum} %Assumption Number
\newcommand{\atheassumpnum}{A\theassumpnum}
\newcommand{\aref}[1]{A\ref{#1}}
\newcounter{goalnum} %Goal Number
\newcommand{\gthegoalnum}{GS\thegoalnum}
\newcommand{\gsref}[1]{GS\ref{#1}}
\newcounter{instnum} %Instance Number
\newcommand{\itheinstnum}{IM\theinstnum}
\newcommand{\iref}[1]{IM\ref{#1}}
\newcounter{reqnum} %Requirement Number
\newcommand{\rthereqnum}{R\thereqnum}
\newcommand{\rref}[1]{R\ref{#1}}
\newcounter{nfrnum} %NFR Number
\newcommand{\rthenfrnum}{NFR\thenfrnum}
\newcommand{\nfrref}[1]{NFR\ref{#1}}
\newcounter{lcnum} %Likely change number
\newcommand{\lthelcnum}{LC\thelcnum}
\newcommand{\lcref}[1]{LC\ref{#1}}
\newcounter{refnum} %Reference Number
\newcommand{\retherefnum}{REF\therefnum}
\newcommand{\reref}[1]{\ref{#1}}
\newcounter{ulcnum} %Unlike likely change number
\newcommand{\ltheulcnum}{ULC\theulcnum}
\newcommand{\ulcref}[1]{ULC\ref{#1}}

\usepackage{fullpage}

\newcommand{\deftheory}[9][None]
{
\newpage
\noindent \rule{\textwidth}{0.5mm}

\paragraph{RefName: } \textbf{#2} \phantomsection 
\label{#2}

\paragraph{Label:} #3

\noindent \rule{\textwidth}{0.5mm}

\paragraph{Equation:}

#4

\paragraph{Description:}

#5

\paragraph{Notes:}

#6

\paragraph{Source:}

#7

\paragraph{Ref.\ By:}

#8

\paragraph{Preconditions for \hyperref[#2]{#2}:}
\label{#2_precond}

#1

\noindent \rule{\textwidth}{0.5mm}

}

\begin{document}

\title{Software Requirements Specification for Bridge Corrosion: A Chloride Exposure Prediction Model} 
\author{Cynthia Liu}
\date{\today}
	
\maketitle

~\newpage

\pagenumbering{roman}

\tableofcontents

~\newpage

\section*{Revision History}

\begin{tabularx}{\textwidth}{p{3cm}p{2cm}X}
\toprule {\bf Date} & {\bf Version} & {\bf Notes}\\
\midrule
Jan 27, 2024 & 1.0 & Initial release\\
\bottomrule
\end{tabularx}

~\newpage

\section{Reference Material}

This section records information for easy reference.

\subsection{Table of Units}

Throughout this document SI (Syst\`{e}me International d'Unit\'{e}s) is employed
as the unit system.  In addition to the basic units, several derived units are
used as described below.  For each unit, the symbol is given followed by a
description of the unit and the SI name.
~\newline

\renewcommand{\arraystretch}{1.2}
%\begin{table}[ht]
  \noindent \begin{tabular}{l l l} 
    \toprule		
    \textbf{symbol} & \textbf{unit} & \textbf{SI}\\
    \midrule 
    \si{\metre} & length & metre\\
    \si{\kilogram} & mass	& kilogram\\
    \si{\second} & time & second\\
    \si{\celsius} & temperature & centigrade\\
    \si{\joule} & energy & joule\\
    \si{\watt} & power & watt (W = \si{\joule\per\second})\\
    \bottomrule
  \end{tabular}
  %	\caption{Provide a caption}
%\end{table}

\plt{Only include the units that your SRS actually uses.}

\plt{Derived units, like newtons, pascal, etc, should show their derivation
    (the units they are derived from) if their constituent units are in the
    table of units (that is, if the units they are derived from are used in the
    document).  For instance, the derivation of pascals as
    $\si{\pascal}=\si{\newton\per\square\meter}$ is shown if newtons and m are
    both in the table.  The derivations of newtons would not be shown if kg and
    s are not both in the table.}

\plt{The symbol for units named after people use capital letters, but the name
  of the unit itself uses lower case.  For instance, pascals use the symbol Pa,
  watts use the symbol W, teslas use the symbol T, newtons use the symbol N,
  etc.  The one exception to this is degree Celsius.  Details on writing metric
  units can be found on the 
  \href{https://www.nist.gov/pml/weights-and-measures/writing-metric-units}
  {NIST} web-page.}


\subsection{Table of Symbols}

The table that follows summarizes the symbols used in this document along with
their units.  The choice of symbols was made to be consistent with the heat
transfer literature and with existing documentation for solar water heating
systems.  The symbols are listed in alphabetical order.

\break
\break

\renewcommand{\arraystretch}{1.2}
%\noindent \begin{tabularx}{1.0\textwidth}{l l X}
\noindent \begin{longtable*}{l l p{12cm}} \toprule
\textbf{symbol} & \textbf{unit} & \textbf{description}\\
\midrule 
$V_{speed}$ & $km/h$ & heavy vehicle speed
\\
$V$ & $miles/h$ & heavy vehicle speed
\\
$b$ & $m$ & tire width 
\\
$WD$ & $m$ & water depth/thickness 
\\ 
$K$ & $m$ & ratio of the tire width that is not a groove to the tire width 
\\ 
$h_{film}$ & $m$ & depth of the water film picked up in each rotation
\\ 
$\rho_{water}$ & $kg/m^{3}$ & density of water 
\\ 
$h_{app}$ & $m$ & daily water film thickness on the road
\\ 
$MR_{CA}$ & $kg/s$ & amount of water displaced by a single tire due to capillary adhesion
\\ 
$MR_{TP}$ & $kg/s$ & amount of water displaced by a single tire due to tread pickup
\\ 
$MR_{BW}$ & $kg/s$ & amount of water displaced by a single tire due to bow
\\ 
$MR_{SW}$ & $kg/s$ & amount of water displaced by a single tire due to side waves
\\ 
$SD_{CA}$ & $kg/m^{3}$ & amount of water in 1 $m^{3}$ volume of air by a single tire due to capillary adhesion
\\ 
$SD_{TP}$ & $kg/m^{3}$ & amount of water in 1 $m^{3}$ volume of air by a single tire due to tread pickup
\\ 
$SD_{BW}$ & $kg/m^{3}$ & amount of water in 1 $m^{3}$ volume of air by a single tire due to bow
\\ 
$SD_{SW}$ & $kg/m^{3}$ & amount of water in 1 $m^{3}$ volume of air by a single tire due to side waves
\\ 
$SD_{total}$ & $kg/m^{3}$ & mass of water per unit air volume kicked up by each passing truck
\\
$t_2$ & days & number of days with snow melting
\\ 
\bottomrule
\end{longtable*}
\plt{Use your problems actual symbols.  The si package is a good idea to use for
  units.}

\subsection{Abbreviations and Acronyms}

\renewcommand{\arraystretch}{1.2}
\begin{tabular}{l l} 
  \toprule		
  \textbf{symbol} & \textbf{description}\\
  \midrule 
  A & Assumption\\
  DD & Data Definition\\
  GD & General Definition\\
  GS & Goal Statement\\
  IM & Instance Model\\
  LC & Likely Change\\
  PS & Physical System Description\\
  R & Requirement\\
  SRS & Software Requirements Specification\\
  \progname{} & \plt{put an expanded version of your program name here (as appropriate)}\\
  TM & Theoretical Model\\
  CA & Capillary Adhesion \\
  TP & Tread Pickup\\
  BW & Bow Waves\\
  SW & Side Waves\\
  AADT & Annual Average Daily Traffic\\
  AADTT & Annual Average Daily Truck Traffic \\

  \bottomrule
\end{tabular}\\

\plt{Add any other abbreviations or acronyms that you add}

\subsection{Mathematical Notation}

\plt{This section is optional, but should be included for projects that make use
  of notation to convey mathematical information.  For instance, if typographic
  conventions (like bold face font) are used to distinguish matrices, this
  should be stated here.  If symbols are used to show mathematical operations,
  these should be summarized here.  In some cases the easiest way to summarize
  the notation is to point to a text or other source that explains the
  notation.}

\plt{This section was added to the template because some students use very
  domain specific notation.  This notation will not be readily understandable to
  people outside of your domain.  It should be explained.}

\newpage

\pagenumbering{arabic}

\plt{This SRS template is based on \citet{SmithAndLai2005, SmithEtAl2007,
  SmithAndKoothoor2016}.  It will get you started.  You should not modify the
  section headings, without first discussing the change with the course
  instructor.  Modification means you are not following the template, which
  loses some of the advantage of a template, especially standardization.
  Although the bits shown below do not include type information, you may need to
  add this information for your problem.  If you are unsure, please can ask the  instructor.}

\plt{Feel free to change the appearance of the report by modifying the LaTeX
  commands.}

\plt{This template document assumes that a single program is being documented.
  If you are documenting a family of models, you should start with a commonality
  analysis.  A separate template is provided for this.  For program
  families you should look at \cite{Smith2006, SmithMcCutchanAndCarette2017}.
  Single family member programs are often programs based on a single physical
  model.  General purpose tools are usually documented as a family.  Families of
  physical models also come up.}

\plt{The SRS is not generally written, or read, sequentially.  The SRS is a
  reference document.  It is generally read in an ad hoc order, as the need
  arises.  For writing an SRS, and for reading one for the first time, the
  suggested order of sections is:
\begin{itemize}
\item Goal Statement
\item Instance Models
\item Requirements
\item Introduction
\item Specific System Description
\end{itemize}
}

\plt{Guiding principles for the SRS document:
\begin{itemize}
\item Do not repeat the same information at the same abstraction level.  If
  information is repeated, the repetition should be at a different abstraction
  level.  For instance, there will be overlap between the scope section and the
  assumptions, but the scope section will not go into as much detail as the
  assumptions section.
\end{itemize}
}

\plt{The template description comments should be disabled before submitting this
  document for grading.}

\plt{You can borrow any wording from the text given in the template.  It is part
  of the template, and not considered an instance of academic integrity.  Of
  course, you need to cite the source of the template.}

\plt{When the documentation is done, it should be possible to trace back to the
  source of every piece of information.  Some information will come from
  external sources, like terminology.  Other information will be derived, like
  General Definitions.}

\plt{An SRS document should have the following qualities: unambiguous,
  consistent, complete, validatable, abstract and traceable.}

\plt{The overall goal of the SRS is that someone that meets the Characteristics
  of the Intended Reader (Section~\ref{sec_IntendedReader}) can learn,
  understand and verify the captured domain knowledge.  They should not have to
  trust the authors of the SRS on any statements.  They should be able to
  independently verify/derive every statement made.}

\section{Introduction}
In Ontario, most of the bridges in highway are made of reinforced concrete (RC) decks. However, the bridges may face the chloride-induced corrosion which damage its surface. There are many elements influencing this situation, one of the most important one is the deicing salts. The primary used is sodium chloride (rock salt),when they melt the snow and in contact with water, they could have a chemical reaction and release the chloride ions. Those chloride could penetrate the concrete and induce corrosion in the reinforcing steel, then damage the bridges’ structure and capacity. \\
There is a tight connection between chloride exposure, weather conditions and traffic flow. Specifically, the amount of deicing salts applied on the road surface greatly depends on the amount of snowfall, and the amount of water and dissolved chloride ions that end up on nearby objects depends on the traffic patterns. This section outlines the document's purpose, delineates its scope of requirements, describes the intended audience's characteristics, and provides an overview of the document's organization.

\plt{The introduction section is written to introduce the problem.  It starts
  general and focuses on the problem domain. The general advice is to start with
a paragraph or two that describes the problem, followed by a ``roadmap''
paragraph.  A roadmap orients the reader by telling them what sub-sections to
expect in the Introduction section.}

\subsection{Purpose of Document}
This document details the requirements of the software Bridge Corrosion. The
responsibilities of the user and software are laid out and the requirements that the software must satisfy are explicitly detailed. This document provides the software requirements specification (SRS) for a project to investigate how the climate and traffic could have impact on the corrosion-induced damage for the reinforced concrete, or to be more specific, how they influence the chloride exposure. 

\plt{This section summarizes the purpose of the SRS document.  It does not focus
  on the problem itself.  The problem is described in the ``Problem
  Description'' section (Section~\ref{Sec_pd}).  The purpose is for the document
  in the context of the project itself, not in the context of this course.
  Although the ``purpose'' of the document is to get a grade, you should not
  mention this.  Instead, ``fake it'' as if this is a real project.  The purpose
  section will be similar between projects.  The purpose of the document is the
  purpose of the SRS, including communication, planning for the design stage,
  etc.}

\subsection{Scope of Requirements} 
The entire document is written as the chloride is the main source of corrosion damage to the reinforced concrete, and chloride ions are transported from the road to the exterior surface of bridge substructures through vehicle spray and splash mechanisms.

\plt{Modelling the real world requires simplification.  The full complexity of
  the actual physics, chemistry, biology is too much for existing models, and
  for existing computational solution techniques.  Rather than say what is in
  the scope, it is usually easier to say what is not.  You can think of it as
  the scope is initially everything, and then it is constrained to create the
  actual scope.  For instance, the problem can be restricted to 2 dimensions, or
  it can ignore the effect of temperature (or pressure) on the material
  properties, etc.}  

\plt{The scope section is related to the assumptions section
  (Section~\ref{sec_assumpt}).  However, the scope and the assumptions are not
  at the same level of abstraction.  The scope is at a high level.  The focus is
  on the ``big picture'' assumptions.  The assumptions section lists, and
  describes, all of the assumptions.}

\plt{The scope section is relevant for later determining typical values of inputs. The scope should make it clear what inputs are reasonable to expect. This is a distinction between scope and context (context is a later section).  Scope affects the inputs while context affects how the software will be used.}

\subsection{Characteristics of Intended Reader} \label{sec_IntendedReader}
Readers of this documentation are expected to have a understanding of high school mathematics and chemistry, as well as the ability to comprehend basic results generated through computational fluid dynamics. The users of Bridge Corrosion may exhibit diverse levels of expertise, as further detailed in Section \ref{SecUserCharacteristics}.

\plt{This section summarizes the skills and knowledge of the readers of the
  SRS.  It does NOT have the same purpose as the ``User Characteristics''
  section (Section~\ref{SecUserCharacteristics}).  The intended readers are the
  people that will read, review and maintain the SRS.  They are the people that
  will conceivably design the software that is intended to meet the
  requirements.  The user, on the other hand, is the person that uses the
  software that is built.  They may never read this SRS document.  Of course,
  the same person could be a ``user'' and an ``intended reader.''}

\plt{The intended reader characteristics should be written as unambiguously and
  as specifically as possible.  Rather than say, the user should have an
  understanding of physics, say what kind of physics and at what level.  For
  instance, is high school physics adequate, or should the reader have had a
  graduate course on advanced quantum mechanics?}

\subsection{Organization of Document}
The organization of this document follows the template for an SRS for scientific computing software proposed by Smith et al. [\reref{ref1}, \reref{ref2}, \reref{ref3}]. Starting with the reference material including units, symbols and abbreviations, this document next introduce the system that we are going to build from general to specific, including the problem, goal, assumptions, theoretical model and instance models. It also talks about the functional and nonfunctional requirements for this project, which could be referred to in process of development.

\plt{This section provides a roadmap of the SRS document.  It will help the
  reader orient themselves.  It will provide direction that will help them
  select which sections they want to read, and in what order.  This section will
  be similar between project.}

\section{General System Description}

This section provides general information about the system.  It identifies the
interfaces between the system and its environment, describes the user
characteristics and lists the system constraints.  \plt{This text can likely be
  borrowed verbatim.}

\plt{The purpose of this section is to provide general information about the
  system so the specific requirements in the next section will be easier to
  understand. The general system description section is designed to be
  changeable independent of changes to the functional requirements documented in
  the specific system description. The general system description provides a
  context for a family of related models.  The general description can stay the
  same, while specific details are changed between family members.}

\subsection{System Context}
\plt{Your system context will include a figure that shows the abstract view of
  the software.  Often in a scientific context, the program can be viewed
  abstractly following the design pattern of Inputs $\rightarrow$ Calculations
  $\rightarrow$ Outputs.  The system context will therefore often follow this
  pattern.  The user provides inputs, the system does the calculations, and then
  provides the outputs to the user.  The figure should not show all of the
  inputs, just an abstract view of the main categories of inputs (like material
  properties, geometry, etc.).  Likewise, the outputs should be presented from
  an abstract point of view.  In some cases the diagram will show other external
  entities, besides the user.  For instance, when the software product is a
  library, the user will be another software program, not an actual end user.
  If there are system constraints that the software must work with external
  libraries, these libraries can also be shown on the System Context diagram.
  They should only be named with a specific library name if this is required by
  the system constraint.}
Figure 1 shows the system context of the software. The user should input a coordinates to the software, and the software will return the predicted chloride exposure over time to the user. The user and the software also assume the following responsibilities.

\begin{figure}[h!]
\begin{center}
 \includegraphics[width=0.8\textwidth]{SystemContextFigure}
\caption{System Context}
\label{Fig_SystemContext} 
\end{center}
\end{figure}

\plt{For each of the entities in the system context diagram its responsibilities
  should be listed.  Whenever possible the system should check for data quality,
  but for some cases the user will need to assume that responsibility.  The list
  of responsibilities should be about the inputs and outputs only, and they
  should be abstract.  Details should not be presented here.  However, the
  information should not be so abstract as to just say ``inputs'' and
  ``outputs''.  A summarizing phrase can be used to characterize the inputs.
  For instance, saying ``material properties'' provides some information, but it
  stays away from the detail of listing every required properties.}

\begin{itemize}
\item User Responsibilities:
\begin{itemize}
\item Provide valid coordinates to the software.
\end{itemize}
\item Bridge Corrosion Responsibilities:
\begin{itemize}
\item Build a database storing the chloride exposure data for every 25 km over time.
\item Search and return the chloride exposure trend at given input coordinate.
\item Provide visualization of the output.
\end{itemize}
\end{itemize}

\plt{Identify in what context the software will typically be used.  Is it for
exploration? education? engineering work? scientific work?. Identify whether it
will be used for mission-critical or safety-critical applications.} \plt{This
additional context information is needed to determine how much effort should be
devoted to the rationale section.  If the application is safety-critical, the
bar is higher.  This is currently less structured, but analogous to, the idea to
the Automotive Safety Integrity Levels (ASILs) that McSCert uses in their
automotive hazard analyses.}

\wss{The }
\subsection{User Characteristics} \label{SecUserCharacteristics}

\plt{This section summarizes the knowledge/skills expected of the user.
  Measuring usability, which is often a required non-function requirement,
  requires knowledge of a typical user.  As mentioned above, the user is a
  different role from the ``intended reader,'' as given in
  Section~\ref{sec_IntendedReader}.  As in Section~\ref{sec_IntendedReader}, the
  user characteristics should be specific an unambiguous.  For instance, ``The
  end user of \progname{} should have an understanding of undergraduate Level 1
  Calculus and Physics.''}
The end user of Bridge Corrosion should have the basic understanding of geographic coordinates. Additionally, users may benefit from some knowledge of bridge construction or civil engineering principles to better understand the context and implications of the predicted chloride exposure.


\subsection{System Constraints}

\plt{System constraints differ from other type of requirements because they
  limit the developers' options in the system design and they identify how the
  eventual system must fit into the world. This is the only place in the SRS
  where design decisions can be specified.  That is, the quality requirement for
  abstraction is relaxed here.  However, system constraints should only be
  included if they are truly required.}
  The software must be able to provide output for coordinates inside Ontario. 
  
\section{Specific System Description}

This section first presents the problem description, which gives a high-level
view of the problem to be solved.  This is followed by the solution characteristics
specification, which presents the assumptions, theories, definitions and finally
the instance models.  \plt{Add any project specific details that are relevant
  for the section overview.}

\subsection{Problem Description} \label{Sec_pd}
This project is intended to investigate how climate, traffic might impact corrosion-induced damage for reinforced concrete bridges by influencing the chloride exposure.

\plt{What problem does your program solve?
The description here should be in the problem space, not the solution space.}

\subsubsection{Terminology and  Definitions}

\plt{This section is expressed in words, not with equations.  It provide the
  meaning of the different words and phrases used in the domain of the problem.
The terminology is used to introduce concepts from the world outside of the
mathematical model  The terminology provides a real world connection to give the
mathematical model meaning.}

This subsection provides a list of terms that are used in the subsequent
sections and their meaning, with the purpose of reducing ambiguity and making it
easier to correctly understand the requirements:

\begin{itemize}

\item Mass flow rate: The amount of water displaced by a single tire 
\item Spray density: The water concentration(mass of water per unit air volume) in the environment
TODO

\end{itemize}

\subsubsection{Physical System Description} \label{sec_phySystDescrip}

\plt{The purpose of this section is to clearly and unambiguously state the
  physical system that is to be modelled. Effective problem solving requires a
  logical and organized approach. The statements on the physical system to be
  studied should cover enough information to solve the problem. The physical
  description involves element identification, where elements are defined as
  independent and separable items of the physical system. Some example elements
  include acceleration due to gravity, the mass of an object, and the size and
  shape of an object. Each element should be identified and labelled, with their
  interesting properties specified clearly. The physical description can also
  include interactions of the elements, such as the following: i) the
  interactions between the elements and their physical environment; ii) the
  interactions between elements; and, iii) the initial or boundary conditions.}

The key physical system of Bridge Corrosion, as shown in Figure 2, simulate the situation that a vehicle spray and splash the water, it includes the following elements:

\begin{itemize}

\item[PS1:] Capillary adhesion: The absorption of water (present on the road surface) by the tires through surface tension.

\item[PS2:] Tread pickup: Water within the grooves of a tire being sprayed and splashed behind the tire by turbulent flow in the grooves.

\item[PS3:] Bow wave: Water sent towards the front of the tire because of the physical displacement of water from the road surface due to the vehicle tires.

\item[PS4:] Side wave: Water sent in the direction perpendicular to the traffic because of the physical displacement of water from the road surface due to the vehicle tires.

\end{itemize}

\plt{A figure here makes sense for most SRS documents}

\begin{figure}[h!]
\begin{center}
\includegraphics[width=0.9\textwidth]{phymodel}
\caption{\label{4mechanism} Mechanisms of vehicle spray and splash}

\end{center}
\end{figure}

% \begin{figure}[h!]
% \begin{center}
% %\rotatebox{-90}
% {
%  \includegraphics[width=0.5\textwidth]{<FigureName>}
% }
% \caption{\label{<Label>} <Caption>}
% \end{center}
% \end{figure}

\subsubsection{Goal Statements}

\plt{The goal statements refine the ``Problem Description''
  (Section~\ref{Sec_pd}).  A goal is a functional objective the system under
  consideration should achieve. Goals provide criteria for sufficient
  completeness of a requirements specification and for requirements
  pertinence. Goals will be refined in Section “Instanced Models”
  (Section~\ref{sec_instance}). Large and complex goals should be decomposed
  into smaller sub-goals.  The goals are written abstractly, with a minimal
  amount of technical language.  They should be understandable by non-domain
  experts.}

\noindent Given the salt application data, climate data and traffic data across different regions and time period , the goal statements are:

\begin{itemize}

\item[GS\refstepcounter{goalnum}\thegoalnum \label{G_ChlorideExposurePrediction}:] Predict the chloride exposure for bridges in Ontario over time.

\item[GS\refstepcounter{goalnum}\thegoalnum \label{G_ExtractData}:] Allow user to input coordinate and return the prediction for the nearest bridge.

\plt{One
    sentence description of the goal.  There may be more than one.  Each Goal
    should have a meaningful label.}

\end{itemize}

\subsection{Solution Characteristics Specification}

\plt{This section specifies the information in the solution domain of the system
  to be developed. This section is intended to express what is required in
  such a way that analysts and stakeholders get a clear picture, and the
  latter will accept it. The purpose of this section is to reduce the problem
  into one expressed in mathematical terms. Mathematical expertise is used to
  extract the essentials from the underlying physical description of the
  problem, and to collect and substantiate all physical data pertinent to the
  problem.}

\plt{This section presents the solution characteristics by successively refining
  models.  It starts with the abstract/general Theoretical Models (TMs) and
  refines them to the concrete/specific Instance Models (IMs).  If necessary
  there are intermediate refinements to General Definitions (GDs).  All of these
  refinements can potentially use Assumptions (A) and Data Definitions (DD).
  TMs are refined to create new models, that are called GMs or IMs. DDs are not
  refined; they are just used. GDs and IMs are derived, or refined, from other
  models. DDs are not derived; they are just given. TMs are also just given, but
  they are refined, not used.  If a potential DD includes a derivation, then
  that means it is refining other models, which would make it a GD or an IM.}

\plt{The above makes a distinction between ``refined'' and ``used.'' A model is
  refined to another model if it is changed by the refinement. When we change a
  general 3D equation to a 2D equation, we are making a refinement, by applying
  the assumption that the third dimension does not matter. If we use a
  definition, like the definition of density, we aren't refining, or changing
  that definition, we are just using it.}

\plt{The same information can be a TM in one problem and a DD in another.  It is
  about how the information is used.  In one problem the definition of
  acceleration can be a TM, in another it would be a DD.}

\plt{There is repetition between the information given in the different chunks
  (TM, GDs etc) with other information in the document.  For instance, the
  meaning of the symbols, the units etc are repeated.  This is so that the
  chunks can stand on their own when being read by a reviewer/user.  It also
  facilitates reuse of the models in a different context.}

\noindent \plt{The relationships between the parts of the document are show in
  the following figure.  In this diagram ``may ref'' has the same role as
  ``uses'' above.  The figure adds ``Likely Changes,'' which are able to
  reference (use) Assumptions.}

\begin{figure}[H]
  \includegraphics[scale=0.9]{RelationsBetweenTM_GD_IM_DD_A.pdf}
\end{figure}

The instance models that govern this project are presented in
Subsection~\ref{sec_instance}.  The information to understand the meaning of the
instance models and their derivation is also presented, so that the instance
models can be verified.

\subsubsection{Types}

\plt{This section is optional. Defining types can make the document easier to
understand.}

\subsection{Scope Decisions}

\plt{This section is optional.}
\subsection{Modelling Decisions}



\plt{This section is optional.}

\subsubsection{Assumptions} \label{sec_assumpt}

\plt{The assumptions are a refinement of the scope.  The scope is general, where
  the assumptions are specific.  All assumptions should be listed, even those
  that domain experts know so well that they are rarely (if ever) written down.}
\plt{The document should not take for granted that the reader knows which
  assumptions have been made. In the case of unusual assumptions, it is
  recommended that the documentation either include, or point to, an explanation
  and justification for the assumption.}
\plt{If it helps with the organization and understandability, the assumptions can be presented as sub sections.  The following sub-sections are options: background theory assumptions, helper theory assumptions, generic theory assumptions, problem specific assumptions, and rationale assumptions}

This section simplifies the original problem and helps in developing the
theoretical model by filling in the missing information for the physical system.
The numbers given in the square brackets refer to the theoretical model [TM],
general definition [GD], data definition [DD], instance model [IM], or likely
change [LC], in which the respective assumption is used.

\begin{itemize}

\item[A\refstepcounter{assumpnum}\theassumpnum \label{A_deicingSalts}:] All the deicing salts are applied on days with snowfall(RefBy: \ddref{dsq})


\item[A\refstepcounter{assumpnum}\theassumpnum \label{A_laneWidth}:] The lane width for all the roads are 3 meters(RefBy: \ddref{dsq})

\item[A\refstepcounter{assumpnum}\theassumpnum \label{A_NaCl}:] The main component of deicing salt is NaCl(RefBy: \ddref{rcl})

\item[A\refstepcounter{assumpnum}\theassumpnum \label{A_AADT}:] Same class of the road over which the bridge spans has the same AADT(RefBy: \dref{csas})

  \plt{Short description of each assumption.  Each assumption
    should have a meaningful label.  Use cross-references to identify the
    appropriate traceability to TM, GD, DD etc., using commands like dref, ddref
    etc.  Each assumption should be atomic - that is, there should not be an
    explicit (or implicit) ``and'' in the text of an assumption.}

\end{itemize}

\subsubsection{Theoretical Models}\label{sec_theoretical}

\plt{Theoretical models are sets of abstract mathematical equations or axioms
  for solving the problem described in Section ``Physical System Description''
  (Section~\ref{sec_phySystDescrip}). Examples of theoretical models are
  physical laws, constitutive equations, relevant conversion factors, etc.}

\plt{Optionally the theory section could be divided into subsections to provide more structure and improve understandability and reusability.  Potential subsections include the following: Context theories, background theories, helper theories, generic theories, problem specific theories, final theories and rationale theories.}

This section focuses on the general equations and laws the project is based
on.  \plt{Modify the examples below for your problem, and add additional models
  as appropriate.}
~\newline
\noindent

%TM1

\noindent
\begin{minipage}{\textwidth}
\renewcommand*{\arraystretch}{1.5}
\begin{tabular}{| p{\colAwidth} | p{\colBwidth}|}
\hline
\rowcolor[gray]{0.9}
Number& TM\refstepcounter{theorynum}\thetheorynum \label{wft}\\
\hline
Label &\bf Water film thickness\\
\hline
% Units&$MLt^{-3}T^0$\\
% \hline
SI Units&\si{m}\\
\hline
Equation& $WD = 6 \times 10^{-4} \cdot T^{0.09} (L \cdot I)^{0.6} \cdot S^{-0.33} $\\
\hline
Description & 
The above equation compute the water film thickness based on the rainfall intensity and pavement surface properties, where
\begin{itemize}

\item $WD$ is the water depth ($m$)

\item $T$ is the texture ($mm$)

\item $L$ is the drainage lengt ($m$)

\item $I$ is the rainfall intensity ($m/h$)

\item $S$ is the slope (ratio)


\end{itemize}

\\
\hline
  Source & Citation here \\
  \hline
  Ref.\ By & \tref{mfrg} \\
  \hline
  Use\ &none \\
  \hline
\end{tabular}

\end{minipage}\\


%TM2
\noindent
\begin{minipage}{\textwidth}
\renewcommand*{\arraystretch}{1.5}
\begin{tabular}{| p{\colAwidth} | p{\colBwidth}|}
\hline
\rowcolor[gray]{0.9}
Number& TM\refstepcounter{theorynum}\thetheorynum \label{mfrg}\\
\hline
Label &\bf Max flow rate general \\
\hline
% Units&$MLt^{-3}T^0$\\
% \hline
SI Units&\si{kg\per s}\\
\hline
Equation& $MR_W = V \cdot b \cdot WD \cdot \rho_{water} $\\

\hline
Description & 
The above equation is the general equation for mass flow rate, which is the maximum amount of water available for splash and spray.
\begin{itemize}

\item $V$ is the truck speed ($km/h$)

\item $b$ is the tire width ($m$)

\item $WD$ is the water depth/thickness ($m$)

\item $\rho_{water}$ is the density of water ($kg/m^{3}$)

\end{itemize}


\\
\hline
  Source & Citation here \\
  \hline
  Ref.\ By & \dref{mfr}\\ 
  \hline
  Use\ & none\\
  \hline
\end{tabular}

\end{minipage}\\

%TM3
\noindent
\begin{minipage}{\textwidth}
\renewcommand*{\arraystretch}{1.5}
\begin{tabular}{| p{\colAwidth} | p{\colBwidth}|}
  \hline
  \rowcolor[gray]{0.9}
  Number& TM\refstepcounter{theorynum}\thetheorynum \label{csasg}\\
  \hline
  Label& \bf Chloride sprayed and splashed \\
\hline
% Units&$MLt^{-3}T^0$\\
% \hline
SI Units&\si{kg\per\metre^3\per vehicle} \\
\hline
Equation & $C_{{s}_{air}} = (SD_{total~cl} \times \frac{1}{\Theta} \times \frac{ADT-N}{N_{lane}}+ SD_{total~cl} \times \frac{N}{N_{lane}}) \times t_{snow}$ \\
  \hline
  Description& Accounting for all the days with snow in a typical winter season, the cumulative mass of chloride ions per unit air volume sprayed and splashed by all the vehicles passing near the bridge pier every winter can be expressed as shown below, where
  
\begin{itemize}

\item $SD_{total~cl}$ is the mass of chloride ions per unit air volume ($kg/m^3/vehicle$)

\item $\Theta$ is the ratio of chloride ions sprayed and splashed by trucks to light-duty vehicles

\item $ADT$ is the average daily traffic

\item $N$ is the fractional number of heavy-duty vehicles out of all vehicles

\item $N_{lane}$ is the number of lanes

\item $t_{snow}$ is the number of days with snow

\end{itemize}


\\
\hline
  Source & Citation here \\
  \hline
  Ref.\ By & \dref{csas} \\ 
  \hline
  Use \ &  \\
  \hline
\end{tabular}
\end{minipage}\\

\plt{``Ref.\ By'' is used repeatedly with the different types of information.
  This stands for Referenced By.  It means that the models, definitions and
  assumptions listed reference the current model, definition or assumption.
  This information is given for traceability.  Ref. By provides a pointer in the
  opposite direction to what we commonly do.  You still need to have a reference
  in the other direction pointing to the current model, definition or
  assumption.  As an example, if TM1 is referenced by GD2, that means that GD2 will
  explicitly include a reference to TM1.}

~\newline

\subsubsection{General Definitions}\label{sec_gendef}

\plt{General Definitions (GDs) are a refinement of one or more TMs, and/or of
  other GDs.  The GDs are less abstract than the TMs.  Generally the reduction
  in abstraction is possible through invoking (using/referencing) Assumptions.
  For instance, the TM could be Newton's Law of Cooling stated abstracting.  The
  GD could take the general law and apply it to get a 1D equation.}

This section collects the laws and equations that will be used in building the
instance models.

\plt{Some projects may not have any content for this section, but the section
  heading should be kept.}  \plt{Modify the examples below for your problem, and
  add additional definitions as appropriate.}

~\newline
%GD1
\noindent
\begin{minipage}{\textwidth}
\renewcommand*{\arraystretch}{1.5}
\begin{tabular}{| p{\colAwidth} | p{\colBwidth}|}
\hline
\rowcolor[gray]{0.9}
Number& GD\refstepcounter{defnum}\thedefnum \label{mfr}\\
\hline
Label &\bf Mass flow rate\\
\hline
% Units&$MLt^{-3}T^0$\\
% \hline
SI Units&\si{kg\per s}\\
\hline
Equation& 
\begin{equation}
     \begin{cases}
     MR_{CA} = V_{speed} \times b \times K \times h_{film} \times \rho_{water} & \text{for} ~ CA \\
      MR_{TP} = V_{speed} \times b \times (1-K) \times h_{app} \times \rho_{water} & \text{for} ~ TP\\
      MR_{BW} = MR_{SW} = 0.5 \times V_{speed} \times b \times (h_{app} \\ - K \times h_{film} - (1-K) \times h_{app}) \times \rho_{water} & \text{for} ~ BW ~ and~ SW \\
      \end{cases}\nonumber
  \end{equation}
  
  \\
\hline
Description & $MR_{CA}, MR_{TP}, MR_{BW}, MR_{SW}$ is the amount of water displaced by a single tire(also called mass flow rate) for capillary adhesion, tread pickup, bow waves and side waves correspondingly ($kg/s$)

\begin{itemize}

\item $V_{speed} $ is the heavy vehicle speed ($km/h$)

\item $b$ is the tire width ($m$)

\item $K$ is the water depth/thickness ($m$)

\item $h_{film}$ is the depth of the water film picked up in each rotation ($m$)

\item $h_{app}$ is the thickness of melted water per day with snow melting ($m$)

\item $\rho_{water}$ is the density of water ($kg/m^{3}$)

\end{itemize}

\\
\hline
  Source & Citation here \\
  \hline
  Ref.\ By & \dref{sd} \\ % \ddref{dsq}, \ddref{dwft}\\
  \hline
  Use\ & \ddref{dwft}\\
  \hline
\end{tabular}

\end{minipage}\\

%GD2
\noindent
\begin{minipage}{\textwidth}
\renewcommand*{\arraystretch}{1.5}
\begin{tabular}{| p{\colAwidth} | p{\colBwidth}|}
\hline
\rowcolor[gray]{0.9}
Number& GD\refstepcounter{defnum}\thedefnum \label{sd}\\
\hline
Label &\bf Spray density \\
\hline
% Units&$MLt^{-3}T^0$\\
% \hline
SI Units&\si{kg\per\metre^3\per vehicle}\\
\hline
Equation&
\begin{equation}
     \begin{cases}
     SD_{CA} = (-2.69 \times 10^{-5} \times V + 2.43 \times 10^{-3}) \times MR_{CA}& \text{for} ~ CA \\
      SD_{TP} = (1.16 \times 10^{-5} \times V - 5.25 \times 10^{-5}) \times MR_{TP} & \text{for} ~ TP\\      
      SD_{BW} = (2.67 \times 10^{-5} \times V - 4.71 \times 10^{-4}) \times MR_{BW} & \text{for} ~ BW\\
       SD_{SW} = (1.65 \times 10^{-5} \times V - 3.99 \times 10^{-4}) \times MR_{SW} & \text{for} ~ SW\\      
      \end{cases}\nonumber
  \end{equation}
\\
\hline
Description & Spray density is derived by conducting computational fluid dynamics simulations on \dref{mfr} to compute the concentration of water kicked up to the environment.  
\begin{itemize}

\item $V$ is the heavy vehicle speed ($miles/h$)

\item $MR_{CA}, MR_{TP}, MR_{BW}, MR_{SW}$ is the mass flow rate for capillary adhesion, tread pickup, bow waves and side waves correspondingly ($kg/s$)
\end{itemize}

\\
\hline
  Source & Citation here \\
  \hline
  Ref.\ By & TODO \\ %\ddref{FluxCoil}, \ddref{FluxPCM}\\
  \hline
  Use \ & \dref{mfr} \\
  \hline
\end{tabular}
\end{minipage}\\

%GD3
\noindent
\begin{minipage}{\textwidth}
\renewcommand*{\arraystretch}{1.5}
\begin{tabular}{| p{\colAwidth} | p{\colBwidth}|}
  \hline
  \rowcolor[gray]{0.9}
  Number& GD\refstepcounter{defnum}\thedefnum \label{csas}\\
  \hline
  Label& \bf Chloride sprayed and splashed \\
\hline
% Units&$MLt^{-3}T^0$\\
% \hline
SI Units&\si{kg\per\metre^3\per vehicle}\\
  \hline
  Equation & $C_{{s}_{air}} = (SD_{total~cl} \times \frac{1}{\Theta} \times (AADT~ per~ lane - AADTT ~per~ lane) + SD_{total~cl} \times AADTT ~per~ lane) \times t_2$ \\
  \hline
  Description& The cumulative mass of chloride ions per unit air volume sprayed and splashed by all vehicles every winter, can be calculated by first finding the mass of chloride ions per unit air volume sprayed and splashed by all the vehicles per day, and times with the number of says with snow melting, where
  
\begin{itemize}

\item $SD_{total~cl}$ is the mass of chloride ions per unit air volume ($kg/m^3/vehicle$)

\item $\Theta$ is the ratio of chloride ions sprayed and splashed by trucks to light-duty vehicles
Theta
\item $AADT ~per~ lane$ is the annual average daily traffic per lane

\item $AADTT~ per~ lane$ is the annual average daily truck traffic per lane

\item $t_2$ is the number of days with snow melting
\end{itemize}

  \\
  \hline
  Sources& Denby, B. R., Sundvor, I., Johansson, C., Pirjola, L., Ketzel, M., Norman, M., ... and Omstedt, G. 2013. “A coupled road dust and surface moisture model to predict non-exhaust road traffic induced particle emissions (NORTRIP). Part 2: Surface moisture and salt impact modelling.” Atmospheric Environment, 81: 485-503, Wang, H., Ranade, R., and Okumus, P. 2022. “Estimating chloride exposure of reinforced concrete bridges using vehicle spray and splash mechanisms.”Structure and Infrastructure Engineering, 1-11.TODO\\
  \hline
  Ref.\ By & \\
  \hline
  Use \ & \ddref{sdtcl}, \tref{csasg} \\
  \hline
\end{tabular}
\end{minipage}\\


\subsubsection*{Detailed derivation of simplified rate of change of temperature}

\plt{This may be necessary when the necessary information does not fit in the
  description field.}
\plt{Derivations are important for justifying a given GD.  You want it to be
  clear where the equation came from.}

\subsubsection{Data Definitions}\label{sec_datadef}



\plt{The Data Definitions are definitions of symbols and equations that are
  given for the problem.  They are not derived; they are simply used by other
  models.  For instance, if a problem depends on density, there may be a data
  definition for the equation defining density.  The DDs are given information
  that you can use in your other modules.}

\plt{All Data Definitions should be used (referenced) by at least one other
  model.}

This section collects and defines all the data needed to build the instance
models. The dimension of each quantity is also given.  \plt{Modify the examples
  below for your problem, and add additional definitions as appropriate.}

~\newline

%DD1, Mapp
\noindent
\begin{minipage}{\textwidth}
\renewcommand*{\arraystretch}{1.5}
\begin{tabular}{| p{\colAwidth} | p{\colBwidth}|}
\hline
\rowcolor[gray]{0.9}
Number& DD\refstepcounter{datadefnum}\thedatadefnum \label{dsq}\\
\hline
Label& \bf Deicing salts quantity\\
\hline
Symbol &$M_{app}$\\
\hline
% Units& $Mt^{-3}$\\
% \hline
  SI Units & $kg/m^2$\\
  \hline
  Equation& 
\begin{equation}
     \begin{cases}
     M_{app} = M_{total}/t_{1} \\
     M_{app}=\frac{V_{salt} \times h_{total}}{t_1 \times W_{lane}}\\
      \end{cases}\nonumber
  \end{equation}\\
  \hline
  Description & The equation determine the quantity of deicing salts applied per day with snowfall, with \aref{A_deicingSalts}. In some case there is absence of the data of $M_{total}$, the second equation is used.
  
\begin{itemize}

\item $M_{total}$ is the total amount of deicing salts applied on the road during the winter season ($kg/m^2$)

\item $t_{1}$ is the number of days with snowfall.

\item $V_{salt}$ is the normalized salt application rate(tonnes/cm/km)

\item $W_{lane}$ is the lane width according to \aref{A_laneWidth}(m).
\end{itemize}

  \\
  \hline
  Sources& Citation here \\
  \hline
  Ref.\ By & \ddref{rsw}   \\
  \hline
\end{tabular}
\end{minipage}\\

%DD2, water thickness
\noindent
\begin{minipage}{\textwidth}
\renewcommand*{\arraystretch}{1.5}
\begin{tabular}{| p{\colAwidth} | p{\colBwidth}|}
\hline
\rowcolor[gray]{0.9}
Number& DD\refstepcounter{datadefnum}\thedatadefnum \label{dwft}\\
\hline
Label& \bf Daily water film thickness\\
\hline
Symbol &$h_{app}$\\
\hline
% Units& $Mt^{-3}$\\
% \hline
  SI Units & \si{\meter}\\
  \hline
  Equation&$h_{app} = h_{total}/t_{snow}$\\
  \hline
  Description & The equaition above calculates the thickness of melted water per day with snow melting.
\begin{itemize}

\item $h_{total}$ is the total water equivalent of the total snowfall during a winter season ($m$)

\item $t_{snow}$ is the number of days with snow melting


\end{itemize}

  \\
  \hline
  Sources& Citation here \\
  \hline
  Ref.\ By & \dref{mfr} TODO\\
  \hline
\end{tabular}
\end{minipage}\\

%DD3, molar mass ratio of chlorine
\noindent
\begin{minipage}{\textwidth}
\renewcommand*{\arraystretch}{1.5}
\begin{tabular}{| p{\colAwidth} | p{\colBwidth}|}
\hline
\rowcolor[gray]{0.9}
Number& DD\refstepcounter{datadefnum}\thedatadefnum \label{rcl}\\
\hline
Label &\bf Ratio of chlorine in deicing salts \\
\hline
% Units&$MLt^{-3}T^0$\\
% \hline
SI Units&none\\
\hline
Equation & $\theta_{chloride}=\frac{\text{mass of }Cl^{-}}{\text{mass of } NaCl}$ \\
\hline
Description & This equation computes the molar mass ratio of chlorine to deicing salts, where
\begin{itemize}
\item $Cl^-$ is the chloride ions whose exposure we want to investigate
\item $NaCl$ is the most commonly used salt
\end{itemize}

\\
\hline
%(https://chem.libretexts.org/Bookshelves/Introductory_Chemistry/Introductory_Chemistry_(CK-12)/04%3A_Atomic_Structure/4.05%3A_Mass_Ratio_Calculation)
  Source & todo\\
  \hline
  Ref.\ By & \ddref{sdtcl} \\ 
  \hline
  Use \ & none \\
  \hline
\end{tabular}
\end{minipage}\\


%DD4, total spray density
\noindent
\begin{minipage}{\textwidth}
\renewcommand*{\arraystretch}{1.5}
\begin{tabular}{| p{\colAwidth} | p{\colBwidth}|}
\hline
\rowcolor[gray]{0.9}
Number& DD\refstepcounter{datadefnum}\thedatadefnum \label{tsd}\\
\hline
Label &\bf Total spray density\\
\hline
% Units&$MLt^{-3}T^0$\\
% \hline
SI Units&\si{kg\per m^3 \per vehicle}\\
\hline
Equation& $SD_{total} = SD_{CA} + SD_{TP} + SD_{BW} + SD_{SW}$\\
\hline
Description & The spray density (i.e. mass of water per unit air volume kicked up by each passing truck), is the sum of the four mechanism.

\begin{itemize}

\item $SD_{CA}$ is the spray density due to capillary adhesion
\item $SD_{TP}$ is the spray density due to tread pickup
\item $SD_{BW}$ is the spray density due to bow waves
\item $SD_{SW}$ is the spray density due to side waves

\end{itemize}

\\
\hline
  Source & Citation here \\
  \hline
  Ref.\ By & \ddref{sdtcl} \\ % \ddref{dsq}, \ddref{dwft}\\
  \hline
  Use\ & \dref{sd}\\
  \hline
\end{tabular}

\end{minipage}\\


%DD5, ratio of salt over water
\noindent
\begin{minipage}{\textwidth}
\renewcommand*{\arraystretch}{1.5}
\begin{tabular}{| p{\colAwidth} | p{\colBwidth}|}
\hline
\rowcolor[gray]{0.9}
Number& DD\refstepcounter{datadefnum}\thedatadefnum \label{rsw}\\
\hline
Label &\bf Ratio of salt over water \\
\hline
% Units&$MLt^{-3}T^0$\\
% \hline
SI Units&none\\
\hline
Equation & $\delta_{salt} =\frac{M_{app}}{h_{app} \times \rho_{water}}$ \\
\hline
Description & This equation computes the ratio of the mass of salt applied per unit area of road to the mass of water per unit area of road, where
\begin{itemize}

\item $M_{app}$ is the quantity of deicing salts applied per day ($kg/m^2$)

\item $h_{app}$ is the thickness of melted water per day ($m$)

\item $\rho_{water}$ is the density of water ($kg/m^{3}$) 
\end{itemize}

\\
\hline
  Source & Citation here \\
  \hline
  Ref.\ By & \ddref{sdtcl} \\ 
  \hline
  Use \ & \ddref{dsq}, \ddref{dwft} \\
  \hline
\end{tabular}
\end{minipage}\\

%DD6, SDtotal cl
\noindent
\begin{minipage}{\textwidth}
\renewcommand*{\arraystretch}{1.5}
\begin{tabular}{| p{\colAwidth} | p{\colBwidth}|}
\hline
\rowcolor[gray]{0.9}
Number& DD\refstepcounter{datadefnum}\thedatadefnum \label{sdtcl}\\
\hline
Label &\bf Mass of chloride ions \\
\hline
% Units&$MLt^{-3}T^0$\\
% \hline
SI Units&\si{kg\per\metre^3\per vehicle} \\
\hline
Equation & $SD_{total ~cl} =SD_{total} \times \delta_{salt} \times \theta_{chloride}$ \\
\hline
Description & This equation computes the mass of chloride ions per unit air volume, where
\begin{itemize}

\item $SD_{total}$ is the mass of water per unit air volume ($kg/m^3/vehicle$)

\item $\delta_{salt}$ is the salt-to-water mass ratio per unit area of road

\item $\theta_{chloride}$ is the molar mass ratio of chlorine to deicing salts

\end{itemize}

\\
\hline
  Source & Citation here \\
  \hline
  Ref.\ By & \tref{csasg} \\ 
  \hline
  Use \ & \ddref{tsd}, \ddref{rsw}, \ddref{rcl} \\
  \hline
\end{tabular}
\end{minipage}\\


\subsubsection{Data Types}\label{sec_datatypes}

\plt{This section is optional.  In many scientific computing programs it isn't
  necessary, since the inputs and outpus are straightforward types, like reals,
  integers, and sequences of reals and integers.  However, for some problems it
  is very helpful to capture the type information.}

\plt{The data types are not derived; they are simply stated and used by other
  models.}

\plt{All data types must be used by at least one of the models.}

\plt{For the mathematical notation for expressing types, the recommendation is
  to use the notation of~\citet{HoffmanAndStrooper1995}.}

This section collects and defines all the data types needed to document the
models. \plt{Modify the examples below for your problem, and add additional
  definitions as appropriate.}

~\newline

\noindent
\begin{minipage}{\textwidth}
\renewcommand*{\arraystretch}{1.5}
\begin{tabular}{| p{\colAwidth} | p{\colBwidth}|}
  \hline
  \rowcolor[gray]{0.9}
  Type Name & Name for Type\\
  \hline
  Type Def & mathematical definition of the type\\
  \hline
  Description & description here
  \\
  \hline
  Sources & Citation here, if the type is borrowed from another source\\
  \hline
\end{tabular}
\end{minipage}\\


\subsubsection{Instance Models} \label{sec_instance}    
\plt{The motivation for this section is to reduce the problem defined in
  ``Physical System Description'' (Section~\ref{sec_phySystDescrip}) to one
  expressed in mathematical terms. The IMs are built by refining the TMs and/or
  GDs.  This section should remain abstract.  The SRS should specify the
  requirements without considering the implementation.}

This section transforms the problem defined in Section~\ref{Sec_pd} into 
one which is expressed in mathematical terms. It uses concrete symbols defined 
in Section~\ref{sec_datadef} to replace the abstract symbols in the models 
identified in Sections~\ref{sec_theoretical} and~\ref{sec_gendef}.

The goals \plt{reference your goals} are solved by \plt{reference your instance
  models}.  \plt{other details, with cross-references where appropriate.}
\plt{Modify the examples below for your problem, and add additional models as
  appropriate.}

~\newline

%IM1
\noindent
\begin{minipage}{\textwidth}
\renewcommand*{\arraystretch}{1.5}
\begin{tabular}{| p{\colAwidth} | p{\colBwidth}|}
  \hline
  \rowcolor[gray]{0.9}
  Number& IM\refstepcounter{instnum}\theinstnum \label{cots}\\
  \hline
  Label& \bf Chloride on the surface \\
  \hline
  Input& $C_{s_{air}}, e, d$\\
  \hline
  Output& $C_s$ \\
  \hline
  Equation& $C_s = 0.015 \times C_{s_{air}} \times e^{-0.05d} + 0.985 \times C_{s_{air}} \times  e^{-0.5d}$\\ 
  \hline
  Description& The above equation computes the chloride ions deposition on bridge substructure, taking into account the distance between the edge of the road near the bridge substructure and the bridge substructure, where
\begin{itemize}

\item $C_{s_{air}}$ is the cumulative mass of chloride ions per unit air volume sprayed and splashed by all the vehicles passing near the bridge pier($kg/m^3$)

\item $d$ is the distance between the road edge and nearby bridge structure($m$)

\item $e$ is the base of natural logarithm.

\end{itemize}

TODO: talk about 0.015 etc

  \\
  \hline
  Sources& \reref{ref3} \\
  \hline
  Ref.\ By & \\
  \hline
\end{tabular}
\end{minipage}\\

%IM2 input coordinate and output the data
\noindent
\begin{minipage}{\textwidth}
\renewcommand*{\arraystretch}{1.5}
\begin{tabular}{| p{\colAwidth} | p{\colBwidth}|}
  \hline
  \rowcolor[gray]{0.9}
  Number& IM\refstepcounter{instnum}\theinstnum \label{ldfsb}\\
  \hline
  Label& \bf Search data for specific coordiinate \\
  \hline
  Input& $(longitude, latitude)$\\
  \hline
  Output& \{$C_{s_1}, C_{s_2}..., C_{s_n}$\} \\
  \hline
  Description & This instance model get the longitude and latitude as input, and return a series of the amount of chloride exposure as output. This is the model that the end user of this software will encounter, where
  \begin{itemize}

\item \{$C_{s_1}, C_{s_2}..., C_{s_n}$\} is a list of chloride exposure data (\{$kg/m^3$\})

\item $(longitude, latitude)$ is the coordinate of a location (\degree, \degree)


\end{itemize}
  \\
  \hline
  Sources& Citation here \\
  \hline
  Ref.\ By & \iref{epcm}\\
  \hline
\end{tabular}
\end{minipage}\\

%~\newline

\subsubsection*{Derivation of ...}

\plt{The derivation shows how the IM is derived from the TMs/GDs.  In cases
  where the derivation cannot be described under the Description field, it will
  be necessary to include this subsection.}

\subsubsection{Input Data Constraints} \label{sec_DataConstraints}    

Table~\ref{TblInputVar} shows the data constraints on the input output
variables.  The column for physical constraints gives the physical limitations
on the range of values that can be taken by the variable.  The column for
software constraints restricts the range of inputs to reasonable values.  The
software constraints will be helpful in the design stage for picking suitable
algorithms.  The constraints are conservative, to give the user of the model the
flexibility to experiment with unusual situations.  The column of typical values
is intended to provide a feel for a common scenario.  The uncertainty column
provides an estimate of the confidence with which the physical quantities can be
measured.  This information would be part of the input if one were performing an
uncertainty quantification exercise. In the Bridge Corrosion project, I would talk about not only the input variables mentioned in instance models, but also include those in other models that the software need to process.\\
%%%%%TODO: read paper
The specification parameters in Table~\ref{TblInputVar} are listed in
Table~\ref{TblSpecParams}.

\begin{table}[!h]
  \caption{Input Variables} \label{TblInputVar}
  \renewcommand{\arraystretch}{1.2}
\noindent \begin{longtable*}{l l l l c} 
  \toprule
  \textbf{Var} & \textbf{Physical Constraints} & \textbf{Software Constraints} &
                             \textbf{Typical Value} & \textbf{Uncertainty}\\
  \midrule 
  %%%%%TODO: read paper
  $TODOOO$ & $L > 0$ & $L_{\text{min}} \leq L \leq L_{\text{max}}$ & 1.5 \si[per-mode=symbol] {\metre} & 10\%
  \\
  \bottomrule
\end{longtable*}
\end{table}

\noindent 
\begin{description}
\item[(*)] \plt{you might need to add some notes or clarifications}
\end{description}

\begin{table}[!h]
\caption{Specification Parameter Values} \label{TblSpecParams}
\renewcommand{\arraystretch}{1.2}
\noindent \begin{longtable*}{l l} 
  \toprule
  \textbf{Var} & \textbf{Value} \\
  \midrule 
  $L_\text{min}$ & 0.1 \si{\metre}\\
  \bottomrule
\end{longtable*}
\end{table}

\subsubsection{Properties of a Correct Solution} \label{sec_CorrectSolution}

\noindent
A correct solution must exhibit the chloride exposure values that does not exceed the solubility limits of chloride ions. 


\plt{fill in the details}.  \plt{These
  properties are in addition to the stated requirements.  There is no need to
  repeat the requirements here.  These additional properties may not exist for
  every problem.  Examples include conservation laws (like conservation of
  energy or mass) and known constraints on outputs, which are usually summarized
  in tabular form.  A sample table is shown in Table~\ref{TblOutputVar}}

\begin{table}[!h]
\caption{Output Variables} \label{TblOutputVar}
\renewcommand{\arraystretch}{1.2}
\noindent \begin{longtable*}{l l} 
  \toprule
  \textbf{Var} & \textbf{Physical Constraints} \\
  \midrule 
  $C_s$ & $C_s < 357 kg/m^3$ (by~%\reref{}TODO)
  \\
  
   \bottomrule
\end{longtable*}
\end{table}

\plt{This section is not for test cases or techniques for verification and
  validation.  Those topics will be addressed in the Verification and Validation
  plan.}

\section{Requirements}

\plt{The requirements refine the goal statement.  They will make heavy use of  references to the instance models.}

This section provides the functional requirements, the business tasks that the
software is expected to complete, and the nonfunctional requirements, the
qualities that the software is expected to exhibit.

\indent 
\newpage
\subsection{Functional Requirements}

\begin{itemize}

\item[R\refstepcounter{reqnum}\thereqnum \label{R_Inputs}:] The user input need to be a coordinate within Ontario.


\plt{Requirements
    for the inputs that are supplied by the user.  This information has to be
    explicit.}

\item[R\refstepcounter{reqnum}\thereqnum \label{R_OutputInputs}:] The output need to be a series of data showing the trend of chloride exposure over time at the input location.
\plt{It isn't
    always required, but often echoing the inputs as part of the output is a
    good idea.}

\item[R\refstepcounter{reqnum}\thereqnum \label{R_Calculate}:] During the calculation, the software should be capable of handling situations where units do not match.
\plt{Calculation
    related requirements.}

\item[R\refstepcounter{reqnum}\thereqnum \label{R_VerifyOutput}:] The output from the previous year should be verifiable against real-world data.
  \plt{Verification related requirements.}

\item[R\refstepcounter{reqnum}\thereqnum \label{R_Output}:] The output should be in two decimal points, showing the mass of chloride ions per unit air volume.
\plt{Output related
    requirements.}

\end{itemize}

\plt{Every IM should map to at least one requirement, but not every requirement
  has to map to a corresponding IM.}

\subsection{Nonfunctional Requirements}

\plt{List your nonfunctional requirements.  You may consider using a fit
  criterion to make them verifiable.}
\plt{The goal is for the nonfunctional requirements to be unambiguous, abstract
  and verifiable.  This isn't easy to show succinctly, so a good strategy may be
to give a ``high level'' view of the requirement, but allow for the details to
be covered in the Verification and Validation document.}
\plt{An absolute requirement on a quality of the system is rarely needed.  For
  instance, an accuracy of 0.0101 \% is likely fine, even if the requirement is
  for 0.01 \% accuracy.  Therefore, the emphasis will often be more on
  describing now well the quality is achieved, through experimentation, and
  possibly theory, rather than meeting some bar that was defined a priori.}
\plt{You do not need an entry for correctness in your NFRs.  The purpose of the
  SRS is to record the requirements that need to be satisfied for correctness.
  Any statement of correctness would just be redundant. Rather than discuss
  correctness, you can characterize how far away from the correct (true)
  solution you are allowed to be.  This is discussed under accuracy.}

\noindent \begin{itemize}

\item[NFR\refstepcounter{nfrnum}\thenfrnum \label{NFR_Reliability}:]   \textbf{Reliability}: The predictions generated by the software should be accurate and reliable, reflecting real-world conditions and factors influencing chloride exposure.
 \plt{Characterize the accuracy by giving the context/use for
    the software.  Maybe something like, ``The accuracy of the computed
    solutions should meet the level needed for $<$engineering or scientific
    application$>$.  The level of accuracy achieved by \progname{} shall be
    described following the procedure given in Section~X of the Verification and
    Validation Plan.''  A link to the VnV plan would be a nice extra.}

\item[NFR\refstepcounter{nfrnum}\thenfrnum \label{NFR_Usability}:] \textbf{Usability}: The software interface should be intuitive and user-friendly, allowing users in the section \ref{SecUserCharacteristics} to easily input coordinates and look at the predicted chloride exposure over time.


  \plt{Characterize the usability by giving the context/use for the software.
    You should likely reference the user characteristics section.  The level of
    usability achieved by the software shall be described following the
    procedure given in Section~X of the Verification and Validation Plan.  A
    link to the VnV plan would be a nice extra.}

\item[NFR\refstepcounter{nfrnum}\thenfrnum \label{NFR_Maintainability}:] \textbf{Maintainability}: The code for this software should be designed and structured in a way that it could be easily comprehended and modified by other potential developers.

\plt{The effort required to make any of the likely
    changes listed for \progname{} should be less than FRACTION of the original
    development time.  FRACTION is then a symbolic constant that can be defined
    at the end of the report.}

\item[NFR\refstepcounter{nfrnum}\thenfrnum \label{NFR_Portability}:]  \textbf{Portability}: This software should be able to run on recent versions of Google Chrome, Firefox, MS Edge and Safari. The operating system include Windows 7+ and Mac OS X 10.7+.
\plt{This NFR is easier to write than the others.  The
    systems that \progname{} should run on should be listed here.  When possible
    the specific versions of the potential operating environments should be
    given.  To make the NFR verifiable a statement could be made that the tests
    from a given section of the VnV plan can be successfully run on all of the
    possible operating environments.}


\item[NFR\refstepcounter{nfrnum}\thenfrnum \label{NFR_Scalability}:]   \textbf{Scalability}: The software should be scalable to accommodate potential future expansions or updates, ensuring its continued usefulness as new data or techniques become available.

\end{itemize}

\subsection{Rationale}

\plt{Provide a rationale for the decisions made in the documentation.  Rationale
should be provided for scope decisions, modelling decisions, assumptions and
typical values.}

\section{Likely Changes}    

\noindent \begin{itemize}

\item[LC\refstepcounter{lcnum}\thelcnum\label{LC_laneWidth}:] The lane width in some area might not be 3 meters, and the lane width standards might change in the future, so \aref{A_laneWidth} is likely to be changed. 
\item[LC\refstepcounter{lcnum}\thelcnum\label{LC_AADT}:] \aref{A_AADT} might be changed with the population density, urbanization, or transportation preferences in different area, which all may influence traffic volume and distribution.

\plt{Give
    the likely changes, with a reference to the related assumption (aref), as appropriate.}
\end{itemize}

\section{Unlikely Changes}    

\noindent \begin{itemize}
\item[ULC\refstepcounter{ulcnum}\theulcnum\label{ULC_saltSame}:] The deicing salt need to be applied on days with snowfall to effectively mitigate the formation of ice and ensure safe road conditions, so \aref{A_deicingSalts} is unlikely to change.

\item[ULC\refstepcounter{ulcnum}\theulcnum\label{ULC_NaCl}:] \aref{A_NaCl} is also unlikely to change, as the main component of deicing salt remain consistent.

 \plt{Give
    the unlikely changes.  The design can assume that the changes listed will
    not occur.}

\end{itemize}

\section{Traceability Matrices and Graphs}

The purpose of the traceability matrices is to provide easy references on what
has to be additionally modified if a certain component is changed.  Every time a
component is changed, the items in the column of that component that are marked
with an ``X'' may have to be modified as well.  Table~\ref{Table:trace} shows the
dependencies of theoretical models, general definitions, data definitions, and
instance models with each other. Table~\ref{Table:R_trace} shows the
dependencies of instance models, requirements, and data constraints on each
other. Table~\ref{Table:A_trace} shows the dependencies of theoretical models,
general definitions, data definitions, instance models, and likely changes on
the assumptions.

\plt{You will have to modify these tables for your problem.}

\plt{The traceability matrix is not generally symmetric.  If GD1 uses A1, that
  means that GD1's derivation or presentation requires invocation of A1.  A1
  does not use GD1.  A1 is ``used by'' GD1.}

\plt{The traceability matrix is challenging to maintain manually.  Please do
  your best.  In the future tools (like Drasil) will make this much easier.}

\afterpage{
\begin{landscape}
\begin{table}[h!]
\centering
\begin{tabular}{|c|c|c|c|c|c|c|c|c|c|c|c|c|c|c|c|c|c|c|c|}
\hline
	& \aref{A_OnlyThermalEnergy}& \aref{A_hcoeff}& \aref{A_mixed}& \aref{A_tpcm}& \aref{A_const_density}& \aref{A_const_C}& \aref{A_Newt_coil}& \aref{A_tcoil}& \aref{A_tlcoil}& \aref{A_Newt_pcm}& \aref{A_charge}& \aref{A_InitTemp}& \aref{A_OpRangePCM}& \aref{A_OpRange}& \aref{A_htank}& \aref{A_int_heat}& \aref{A_vpcm}& \aref{A_PCM_state}& \aref{A_Pressure} \\
\hline
\tref{T_COE}        & X& & & & & & & & & & & & & & & & & & \\ \hline
\tref{T_SHE}        & & & & & & & & & & & & & & & & & & & \\ \hline
\tref{T_LHE}        & & & & & & & & & & & & & & & & & & & \\ \hline
\dref{NL}           & & X& & & & & & & & & & & & & & & & & \\ \hline
\dref{ROCT}         & & & X& X& X& X& & & & & & & & & & & & & \\ \hline
\ddref{FluxCoil}    & & & & & & & X& X& X& & & & & & & & & & \\ \hline
\ddref{FluxPCM}     & & & X& X& & & & & & X& & & & & & & & & \\ \hline
\ddref{D_HOF}       & & & & & & & & & & & & & & & & & & & \\ \hline
\ddref{D_MF}        & & & & & & & & & & & & & & & & & & & \\ \hline
\iref{ewat}         & & & & & & & & & & & X& X& & X& X& X& & & X \\ \hline
\iref{epcm}         & & & & & & & & & & & & X& X& & & X& X& X& \\ \hline
\iref{I_HWAT}       & & & & & & & & & & & & & & X& & & & & X \\ \hline
\iref{I_HPCM}       & & & & & & & & & & & & & X& & & & & X & \\ \hline
\lcref{LC_tpcm}     & & & & X& & & & & & & & & & & & & & & \\ \hline
\lcref{LC_tcoil}    & & & & & & & & X& & & & & & & & & & & \\ \hline
\lcref{LC_tlcoil}   & & & & & & & & & X& & & & & & & & & & \\ \hline
\lcref{LC_charge}   & & & & & & & & & & & X& & & & & & & & \\ \hline
\lcref{LC_InitTemp} & & & & & & & & & & & & X& & & & & & & \\ \hline
\lcref{LC_htank}    & & & & & & & & & & & & & & & X& & & & \\
\hline
\end{tabular}
\caption{Traceability Matrix Showing the Connections Between Assumptions and Other Items}
\label{Table:A_trace}
\end{table}
\end{landscape}
}

\begin{table}[h!]
\centering
\begin{tabular}{|c|c|c|c|c|c|c|c|c|c|c|c|c|c|c|c|c|c|c|c|c|c|c|c|}
\hline        
	& \tref{T_COE}& \tref{T_SHE}& \tref{T_LHE}& \dref{NL}& \dref{ROCT} & \ddref{FluxCoil}& \ddref{FluxPCM} & \ddref{D_HOF}& \ddref{D_MF}& \iref{ewat}& \iref{epcm}& \iref{I_HWAT}& \iref{I_HPCM} \\
\hline
\tref{T_COE}     & & & & & & & & & & & & & \\ \hline
\tref{T_SHE}     & & & X& & & & & & & & & & \\ \hline
\tref{T_LHE}     & & & & & & & & & & & & & \\ \hline
\dref{NL}        & & & & & & & & & & & & & \\ \hline
\dref{ROCT}      & X& & & & & & & & & & & & \\ \hline
\ddref{FluxCoil} & & & & X& & & & & & & & & \\ \hline
\ddref{FluxPCM}  & & & & X& & & & & & & & & \\ \hline
\ddref{D_HOF}    & & & & & & & & & & & & & \\ \hline
\ddref{D_MF}     & & & & & & & & X& & & & & \\ \hline
\iref{ewat}      & & & & & X& X& X& & & & X& & \\ \hline
\iref{epcm}      & & & & & X& & X& & X& X& & & X \\ \hline
\iref{I_HWAT}    & & X& & & & & & & & & & & \\ \hline
\iref{I_HPCM}    & & X& X& & & & X& X& X& & X& & \\
\hline
\end{tabular}
\caption{Traceability Matrix Showing the Connections Between Items of Different Sections}
\label{Table:trace}
\end{table}

\begin{table}[h!]
\centering
\begin{tabular}{|c|c|c|c|c|c|c|c|}
\hline
	& \iref{ewat}& \iref{epcm}& \iref{I_HWAT}& \iref{I_HPCM}& \ref{sec_DataConstraints}& \rref{R_RawInputs}& \rref{R_MassInputs} \\
\hline
\iref{ewat}            & & X& & & & X& X \\ \hline
\iref{epcm}            & X& & & X& & X& X \\ \hline
\iref{I_HWAT}          & & & & & & X& X \\ \hline
\iref{I_HPCM}          & & X& & & & X& X \\ \hline
\rref{R_RawInputs}     & & & & & & & \\ \hline
\rref{R_MassInputs}    & & & & & & X& \\ \hline
\rref{R_CheckInputs}   & & & & & X& & \\ \hline
\rref{R_OutputInputs}  & X& X& & & & X& X \\ \hline
\rref{R_TempWater}     & X& & & & & & \\ \hline 
\rref{R_TempPCM}       & & X& & & & & \\ \hline
\rref{R_EnergyWater}   & & & X& & & & \\ \hline
\rref{R_EnergyPCM}     & & & & X& & & \\ \hline
\rref{R_VerifyOutput}  & & & X& X& & & \\ \hline
\rref{R_timeMeltBegin} & & X& & & & & \\ \hline
\rref{R_timeMeltEnd}   & & X& & & & & \\ 
\hline
\end{tabular}
\caption{Traceability Matrix Showing the Connections Between Requirements and Instance Models}
\label{Table:R_trace}
\end{table}

The purpose of the traceability graphs is also to provide easy references on
what has to be additionally modified if a certain component is changed.  The
arrows in the graphs represent dependencies. The component at the tail of an
arrow is depended on by the component at the head of that arrow. Therefore, if a
component is changed, the components that it points to should also be
changed. Figure~\ref{Fig_ATrace} shows the dependencies of theoretical models,
general definitions, data definitions, instance models, likely changes, and
assumptions on each other. Figure~\ref{Fig_RTrace} shows the dependencies of
instance models, requirements, and data constraints on each other.

% \begin{figure}[h!]
% 	\begin{center}
% 		%\rotatebox{-90}
% 		{
% 			\includegraphics[width=\textwidth]{ATrace.png}
% 		}
% 		\caption{\label{Fig_ATrace} Traceability Matrix Showing the Connections Between Items of Different Sections}
% 	\end{center}
% \end{figure}


% \begin{figure}[h!]
% 	\begin{center}
% 		%\rotatebox{-90}
% 		{
% 			\includegraphics[width=0.7\textwidth]{RTrace.png}
% 		}
% 		\caption{\label{Fig_RTrace} Traceability Matrix Showing the Connections Between Requirements, Instance Models, and Data Constraints}
% 	\end{center}
% \end{figure}

\section{Development Plan}

\plt{This section is optional.  It is used to explain the plan for developing
  the software.  In particular, this section gives a list of the order in which
  the requirements will be implemented.  In the context of a course  this is
  where you can indicate which requirements will be implemented as part of the
  course, and which will be ``faked'' as future work.  This section can be
  organized as a prioritized list of requirements, or it could should the
  requirements that will be implemented for ``phase 1'', ``phase 2'', etc.}

\section{Values of Auxiliary Constants}

\plt{Show the values of the symbolic parameters introduced in the report.}

\plt{The definition of the requirements will likely call for SYMBOLIC\_CONSTANTS.
Their values are defined in this section for easy maintenance.}

\plt{The value of FRACTION, for the Maintainability NFR would be given here.}

\newpage

\bibliographystyle {plainnat}
\bibliography {../../refs/References}

\newpage

\noindent \plt{The following is not part of the template, just some things to consider
  when filing in the template.}

\noindent \plt{Grammar, flow and \LaTeX advice:
\begin{itemize}
\item For Mac users \texttt{*.DS\_Store} should be in \texttt{.gitignore}
\item \LaTeX{} and formatting rules
\begin{itemize}
\item Variables are italic, everything else not, includes subscripts (link to
  document)
\begin{itemize}
\item \href{https://physics.nist.gov/cuu/pdf/typefaces.pdf}{Conventions}
\item Watch out for implied multiplication
\end{itemize}
\item Use BibTeX
\item Use cross-referencing
\end{itemize}
\item Grammar and writing rules
\begin{itemize}
\item Acronyms expanded on first usage (not just in table of acronyms)
\item ``In order to'' should be ``to''
\end{itemize}
\end{itemize}}

\noindent \plt{Advice on using the template:
\begin{itemize}
\item Difference between physical and software constraints
\item Properties of a correct solution means \emph{additional} properties, not
  a restating of the requirements (may be ``not applicable'' for your problem).
  If you have a table of output constraints, then these are properties of a
  correct solution.
\item Assumptions have to be invoked somewhere
\item ``Referenced by'' implies that there is an explicit reference
\item Think of traceability matrix, list of assumption invocations and list of
  reference by fields as automatically generatable
\item If you say the format of the output (plot, table etc), then your
  requirement could be more abstract
\end{itemize}
}

\newpage{}
\section*{Appendix --- Reflection}

The information in this section will be used to evaluate the team members on the
graduate attribute of Lifelong Learning.  Please answer the following questions:

\begin{enumerate}
  \item Which of the courses you have taken, or are currently taking, will help
  your team to be successful with your capstone project.
  \item What knowledge and skills will the team collectively need to acquire to
  successfully complete this capstone project?  Examples of possible knowledge
  to acquire include domain specific knowledge from the domain of your
  application, or software engineering knowledge, mechatronics knowledge or
  computer science knowledge.  Skills may be related to technology, or writing,
  or presentation, or team management, etc.  You should look to identify at
  least one item for each team member.
  \item For each of the knowledge areas and skills identified in the previous
  question, what are at least two approaches to acquiring the knowledge or
  mastering the skill?  Of the identified approaches, which will each team
  member pursue, and why did they make this choice?
\end{enumerate}

\end{document}